\documentclass[aspectratio=169]{beamer}
\usepackage{etex} % fixes new-dimension error
\usepackage{lmodern}
\usepackage[T1]{fontenc}

\usepackage{graphicx,amsmath}
\usepackage{stmaryrd} % cf. interleave
\usepackage{booktabs}
\usepackage{amscd}
\usepackage{multicol}
\usepackage[absolute,overlay]{textpos}
\usepackage{alltt}
\usepackage{proof}
\usepackage{subcaption}
\usepackage{tikz}
\usepackage{tikz-cd}
\usepackage[new]{old-arrows}
\usepackage[all]{xy}
\usepackage{pgfplots}
\usepackage{textcomp}
\usepackage{cancel}

%------ using color ---------------------------------------------------------
%\newrgbcolor{goldenrod}{.80392 .60784 .11373}
%\newrgbcolor{darkgoldenrod}{.5451 .39608 .03137}
%\newrgbcolor{brown}{.15 .15 .15}
%\newrgbcolor{darkolivegreen}{.33333 .41961 .18431}
\definecolor{goldenrod}{rgb}{.80392,.60784,.11373}
\definecolor{darkgoldenrod}{rgb}{.5451,.39608,.03137}
\definecolor{brown}{rgb}{.15,.15,.15}
\definecolor{darkolivegreen}{rgb}{.33333,.41961,.18431}
\definecolor{darkgreen}{rgb}{0,0.6,0}
\definecolor{myGray}{gray}{0.85}
%


% \newcommand{\red}[1]{\textcolor{red!80!black}{#1}\xspace}
% \newcommand{\blue}[1]{\textcolor{blue}{#1}\xspace}
% \newcommand{\gold}[1]{\textcolor{darkgoldenrod}{#1}\xspace}
\newcommand{\grey}[1]{\textcolor{myGray}{#1}\xspace}
\def\laplace#1#2{*\txt{\mbox{ \fcolorbox{black}{myGray}{$\begin{array}{c}\mbox{#1}\\\\#2\\\\\end{array}$} }}}
\newenvironment{bluein}{\blue}{\black\hskip -2.5pt}

%------ contexts  ---------------------------------------------------------

\newtheorem{defi}{Defini\cao}[section]
\newtheorem{defi*}{Defini\cao}
\newtheorem{lema*}{Lema}
\newenvironment{lsbcom}
      {\footnotesize  \hrule ~\\ ~\\ {\bf \sc Nota:} }
      {\hrule  ~\\ ~\\  \normalsize}
      
\newenvironment{lsbcomi}
      {\footnotesize  \hrule ~\\ ~\\ {\bf \sc Note:} }
      {\hrule  ~\\ ~\\  \normalsize}
      
\newenvironment{demo}%
     {\vspace{-5mm}\noindent {\bf Prova:}}%
    {\par \nopagebreak  \noindent \fimdemo \vspace{3mm} }
\newenvironment{demoi}%
     {\vspace{-5mm}\noindent {\bf Proof:}}%
    {\par \nopagebreak  \noindent \fimdemo \vspace{3mm} }


% \newcommand{\N}{\ensuremath{\mathbb{N}}\xspace}
% \newcommand{\Z}{\ensuremath{\mathbb{Z}}\xspace}
% \newcommand{\R}{\ensuremath{\mathbb{R}}\xspace}

\newcommand{\cR}{\ensuremath{\mathcal{R}}\xspace}

% \newcommand{\set}[1]{\left\{ #1 \right\}} % {a,b,...z}
\def\bang{{!}}
\def\deff{\stackrel{\rm def}{=}}          % Function definition symbol
%\def\deff{\, :=\, }          % Function definition symbol


\def\always{\boxempty}
\def\eventual{\Diamond}
\def\nexts{\bigcirc}
\def\until{\mathbin{\mathcal U}}




%------ CCS          -----------------------------------------%
\def\ppe{\mathbin{\vartriangleright}}
\def\kcomp{\mathbin{\boldsymbol{\bullet}}}
%\def\qcomp{\mathbin{\boldsymbol{;}}}
\def\ssp{\textsc{skip}}
\def\fim{\dagger}
%\def\ppe{\gg}
\def\ff{\ensuremath{\mi{f\!f}}\xspace}
\def\tt{\ensuremath{\mi{t\!t}}\xspace}
\def\cnil{\mathbf{0}}
\def\cpf#1#2{#1 . #2}                           % a.P
\def\cou#1#2{#1 \mathbin{+} #2}                 % P + Q
%\def\crt#1#2{\mathbin{#1 \setminus_{#2}}}       % P \ A
%\def\crtt#1#2{\mathbin{#1 \setminus\!\setminus_{#2}}}       % P \ A
%\def\crt#1#2{\mathsf{new}\, #2\;  #1}       % P \ A
\def\crt#1#2{#1 \backslash #2}       % P \ A
%\def\crn#1#2{\{#2\}\, #1}                  % P[f]

\def\crn#1#2{\mathbin{#1[#2]}}                  % P[f]
\def\couit#1#2{\Sigma_{#1}#2}                  %  + i=1,n
\def\cpar#1#2{#1 \mid #2}                       %  |
\def\ctpar#1#2{#1 \parallel #2}                       %  |
\def\cpars#1#2#3{#1 \mid_{#3} #2}               %  |S
\def\ffix#1#2{\underline{fix}~(#1\, =\, #2)}  % fix X
\def\fffix#1#2#3{\underline{fix}_{#1}~(#2\, =\, #3)}  % fix X
\def\tfix#1#2{\underline{\Tilde{fix}}~(\Tilde{#1}\, =\, \Tilde{#2})}  % fix X
%\def\ainv#1{\Bar{#1}}                   % ~ a
\def\ainv#1{\overline{#1}}                   % ~ a
\def\cif#1#2{\fuc{if}\, #1\, \fuc{then}\, #2}
\def\ccif#1#2#3{\fuc{if}\, #1\, \fuc{then}\, #2\, \fuc{else}\, #3}

\def\fres#1#2{#1 \restriction #2}                

\def\mean#1{\mathopen{[\![}#1\mathclose{]\!]}}
\def\llbracket{\mathopen{[\![}}
\def\rrbracket{\mathopen{]\!]}}

\def\cfree#1{\fuc{fn} (#1)}
\def\cbound#1{\fuc{bn} (#1)}
\def\anew#1{\overline{#1}\mathsf{new}\, } 
\def\transitatau{\rtran{\tau}}
\def\transitaa{\rtran{a}}


% 2015




%%%%%%%%%%%%%%%% NUNO reconf


\def\trans#1{\stackrel{#1}{\longrightarrow}}
\def\TS#1{\mathcal{G}(#1)}
\def\TSn#1{\mathcal{G}_{nodes}(#1)}




%--------------------
%--- by jose 2016 ---
%--------------------

\newcommand{\myblock}[1]{\begin{beamercolorbox}[dp=1ex,center,rounded=true]%
  {postit} {\large \textbf{#1}} \end{beamercolorbox}}%
\def\trans#1{\xrightarrow{#1}}  % - a - > 
\def\Trans#1{\stackrel{#1}{\Longrightarrow}} % =a=> 
\def\transtau{\xrightarrow{\tau}}  % - a - > 


%\newcommand{\evm}[1]{\langle #1 \rangle\,\fi}
%\newcommand{\alm}[1]{[#1]\,}
\newcommand{\evm}[1]{\if\relax\detokenize{#1}\relax 
  \Diamond \else\langle #1 \rangle\fi}
\newcommand{\alm}[1]{\if\relax\detokenize{#1}\relax 
  \boxempty \else[#1]\fi}

\newcommand{\myparagraph}[1]{\medskip\noindent\textbf{#1}~~}



% Listing
\lstset{
  language=Scala,
  basicstyle=\ttfamily\footnotesize, % overriding size
  commentstyle=\sffamily\color{green!60!black},
}
% \lstset{%language=Java
% %  ,basicstyle=\footnotesize
%   ,columns=fullflexible %space-fexible
%   ,keepspaces
% %  ,numberstyle=\tiny
%   ,mathescape=true
%   ,showstringspaces=false
% %  ,morekeywords={refract,global,local,on-change}
%   ,morecomment=[l]{\%}
%   ,commentstyle=\sl\sffamily\color{gray}\scriptsize
%   ,basicstyle=\ttfamily\relsize{-0.5}
%   ,keywordstyle=\bf\sffamily\color{purple}
% %  ,emphstyle=\it\sffamily\color{blue!80!black}
%   ,emphstyle=\bfseries\itshape\color{blue!80!black}
%   ,emphstyle={[2]\itshape\color{red!70!black}}%\underbar}
%   ,stringstyle=\color{darkgreen}
%   ,alsoletter={-,||,+,<>,&&,=>}
%   ,literate=*{->}{{{\color{red!70!black}$\to$}}}{1}
%              {.}{{{\color{red!70!black}.}}}{1}
%              {+}{{{\color{red!70!black}+}}}{1}
%              {*}{{{\color{red!70!black}*}}}{1}
%              {\#}{{{\color{red!70!black}\#}}}{1}
%   ,emph={act,proc,init,sort}
%   ,emph={[2]block,hide,comm,rename,allow,||,<>,sum,&&,=>}
% %  ,emphstyle={[2]\color{blue}}2
%   ,framerule=1pt
%   ,backgroundcolor=\color{black!2}
%   ,rulecolor=\color{black!30}
%   ,frame=tblr
%   ,xleftmargin=4pt
%   ,xrightmargin=4pt
%   ,captionpos=b
% %  ,belowcaptionskip=\medskipamount
%   ,aboveskip=\baselineskip
%   ,floatplacement=htb
% }
\lstdefinestyle{bash}{literate=*}
% \newcommand{\code}[1]{\lstinline[basicstyle=\ttfamily\relsize{-0.5},keywordstyle=\bf\sffamily\color{purple},columns=fullflexible,keepspaces]�#1�}
\newcommand{\code}[1]{\lstinline[columns=fullflexible,keepspaces]�#1�}
\newcommand{\mcode}[1]{\text{\code{#1}}}
\newcommand{\bash}[1]{\lstinline[basicstyle=\ttfamily\relsize{-0.5}\color{darkgreen},keywordstyle=\bf\sffamily\color{purple},columns=fullflexible,keepspaces,literate=*]�#1�}


%%%%%
% Named environments (no counters) 
\newenvironment{theorem}[2][Theorem]{\begin{trivlist}
\item[\hskip \labelsep {\bfseries #1}\hskip \labelsep {\bfseries #2.}]}{\end{trivlist}}
\newenvironment{lemma}[2][Lemma]{\begin{trivlist}
\item[\hskip \labelsep {\bfseries #1}\hskip \labelsep {\bfseries #2.}]}{\end{trivlist}}
%\newenvironment{exercise}[2][Exercise]{\begin{trivlist}
%\item[\hskip \labelsep {\bfseries #1}\hskip \labelsep {\bfseries #2.}]}{\end{trivlist}}
\newenvironment{problem}[2][Problem]{\begin{trivlist}
\item[\hskip \labelsep {\bfseries #1}\hskip \labelsep {\bfseries #2.}]}{\end{trivlist}}
\newenvironment{question}[2][Question]{\begin{trivlist}
\item[\hskip \labelsep {\bfseries #1}\hskip \labelsep {\bfseries #2.}]}{\end{trivlist}}
\newenvironment{corollary}[2][Corollary]{\begin{trivlist}
\item[\hskip \labelsep {\bfseries #1}\hskip \labelsep {\bfseries #2.}]}{\end{trivlist}}
 
% Environments with counters
\newtheoremstyle{myplain} {8mm}% (Space above)
{3mm}% (Space below)
{}% (Body font)
{}% (Indent amount)
{\bfseries\large}% (Theorem head font)
{.}% (Punctuation after theorem head)
{.5em}% (Space after theorem head)2
{}% (Theorem head spec (can be left empty, meaning �normal�))

\theoremstyle{myplain}
\newtheorem{myExercise}{Exercise}

\theoremstyle{definition} % no italics
\newtheorem{subexercise}{}[myExercise]

\newcommand{\ex}[1]{\begin{myExercise}#1\end{myExercise}}
\newcommand{\subex}[1]{\begin{subexercise}#1\end{subexercise}}

\newcommand{\mytikz}[2][]{\medskip\centerline{\begin{tikzpicture}[#1]#2\end{tikzpicture}}\medskip}



%%%%% REO %%%%
\pgfdeclarelayer{background}
\pgfdeclarelayer{threadground}
\pgfsetlayers{background,threadground,main}

\usepackage{pgfplots}

% thickness of the lines
\tikzstyle{border}   = [thick]
\tikzstyle{reodist}    = [node distance=15mm]

% nodes and I/O
\tikzstyle{reonode}  = [border,circle,inner sep=1.5pt,reodist]
\tikzstyle{mixed}    = [reonode, line width=0, %draw=black,
%                        outer color=black,inner color=black!20
                        shading=ball,ball color=black]
\tikzstyle{boundary} = [reonode,draw=black,fill=white]
\tikzstyle{point}    = [reonode,draw=black,fill=black,inner sep=0.5pt]
\tikzstyle{io}       = [border,rectangle,draw=black,fill=white,inner sep=3.25pt,node distance=0.75cm]
\tikzstyle{ioblack}  = [border,rounded corners=0,rectangle,draw=black,
    fill=black,inner sep=3.25pt,node distance=0.75cm]
\tikzstyle{comp}     = [border,rectangle,draw=black,fill=white,inner sep=3.25pt,reodist]
\tikzstyle{token} = [inner sep=0.8mm,regular polygon,regular polygon sides=5,draw=black, fill=#1]
\newcommand{\token}{\tikz \node[token=green!70!black] {};\xspace}


%% animations
\tikzstyle{animflow}=[blue,draw opacity=0.2,line width=2.7mm,line cap=round,line join=round]                   
\tikzstyle{animnf}=[postaction=decorate,line width=0,draw opacity=0,
                  decoration={markings,mark=at position #1 with 
                  {\node[regular polygon,regular polygon sides=3,rotate=30,draw=red,draw opacity=0.5,line width=1.5pt,line cap=round,line join=round, rounded corners=0.5pt,
                  transform shape,inner sep=1.3pt]{};}}]
\tikzstyle{animfflow}=[blue!20,line width=2.7mm,line cap=round,line join=round]                   


% channels
\tikzstyle{channel}=[border,>=stealth]
\tikzstyle{sync}=[channel,->]
\tikzstyle{lossy}=[channel,->,dashed]
\tikzstyle{sdrain}=[channel,>-<]
\tikzstyle{sspout}=[channel,<->]
\tikzstyle{fifo}=[channel,->,
                  postaction=decorate,
                  decoration={markings,mark=at position 0.5 with 
                  {\node[rectangle,draw=black,fill=white,rounded corners=0,
                  transform shape,minimum width=6mm]{#1};}}]
\tikzstyle{fifos}=[channel,->,
                  postaction=decorate,
                  decoration={markings,mark=at position 0.5 with 
                  {\node[rectangle,draw=black,fill=white,rounded corners=0,
                  transform shape,minimum width=4mm]{#1};}}]
\tikzstyle{fifocol}=[channel,->,
                  postaction=decorate,
                  decoration={markings,mark=at position 0.5 with 
                  {\node[rectangle,draw=black,fill=#1,rounded corners=0,
                  transform shape,minimum width=6mm]{};}}]
\tikzstyle{vare}=[channel,->,
                  postaction=decorate,
                  decoration={markings,mark=at position 0.5 with 
                  {\node[rectangle,draw=black,fill=white,rounded corners=3,
                  transform shape,minimum width=6mm]{#1};}}]
\tikzstyle{varf}=[channel,->,
                  postaction=decorate,
                  decoration={markings,mark=at position 0.5 with 
                  {\node[rectangle,draw=black,fill=black,rounded corners=3,
                  transform shape,minimum width=6mm]{#1};}}]
\tikzstyle{filter}=[channel,->,
                  decorate,rounded corners=0,
                  decoration={zigzag,segment length=1.3mm, 
                  pre length=4mm,post length=4mm,#1}]
\tikzstyle{lfilter}=[channel,->,
                  decorate,rounded corners=0,
                  decoration={zigzag,segment length=1.3mm, 
                  pre length=#1,post length=#1}]
\tikzstyle{llfilter}=[channel,->,
                  decorate,rounded corners=0,
                  decoration={zigzag,segment length=1.3mm, 
                  pre length=#1,post length=#1}]
\tikzstyle{transf}=[channel,->,
                  postaction=decorate,
                  decoration={markings,mark=at position 0.45 with 
                  {\node[isosceles triangle,draw=black,fill=white,rounded corners=0,
                  transform shape,inner sep=2pt,#1]{};}}]
\tikzstyle{clossy}=[channel,->,dashed,
                  postaction=decorate,
                  decoration={markings,mark=at position 0.5 with 
                  {\node[sloped]{!};}}]
\tikzstyle{pdrain}=[channel,>-<,
                  postaction=decorate,
                  decoration={markings,
                    mark=at position 0.5 with
                      {\draw[channel,-]
                        (-2pt,-3pt) -- (-2pt,3pt) [transform shape]
                        (2pt,-3pt) -- (2pt,3pt) [transform shape];},
                    mark=at position 0.25 with {\node[sloped]{!};}}]

\tikzstyle{adrain}=[channel,>-<,
    postaction=decorate,
    decoration={markings,mark=at position 0.5 with
    {\draw[channel,-] (-2pt,-3pt) -- (-2pt,3pt) [transform shape]
                      (2pt,-3pt) -- (2pt,3pt) [transform shape];}}]
\tikzstyle{aspout}=[channel,<->,
    postaction=decorate,
    decoration={markings,mark=at position 0.5 with
    {\draw[channel,-] (-2pt,-3pt) -- (-2pt,3pt) [transform shape]
                      (2pt,-3pt) -- (2pt,3pt) [transform shape];}}]


% connector box
\tikzstyle{connector}=[fill=black!10,rounded corners]

\newcommand{\reoconnector}[2][]{
  \begin{tikzpicture}[line join=round,#1]
  #2
  \end{tikzpicture}
}

%%% LOTS OF WORK for an exclusive router.

% \pgfdeclareshape{circle cross}
% {
%   \inheritsavedanchors[from=circle] % this is nearly a circle
%   \inheritanchorborder[from=circle]
%   \inheritanchor[from=circle]{north}
%   \inheritanchor[from=circle]{north west}
%   \inheritanchor[from=circle]{north east}
%   \inheritanchor[from=circle]{center}
%   \inheritanchor[from=circle]{west}
%   \inheritanchor[from=circle]{east}
%   \inheritanchor[from=circle]{mid}
%   \inheritanchor[from=circle]{mid west}
%   \inheritanchor[from=circle]{mid east}
%   \inheritanchor[from=circle]{base}
%   \inheritanchor[from=circle]{base west}
%   \inheritanchor[from=circle]{base east}
%   \inheritanchor[from=circle]{south}
%   \inheritanchor[from=circle]{south west}
%   \inheritanchor[from=circle]{south east}
%   \inheritbackgroundpath[from=circle]
%   \foregroundpath{
%     \centerpoint%
%     \pgf@xc=\pgf@x%
%     \pgf@yc=\pgf@y%
%     \pgfutil@tempdima=\radius%
%     \pgfmathsetlength{\pgf@xb}{\pgfkeysvalueof{/pgf/outer xsep}}%  
%     \pgfmathsetlength{\pgf@yb}{\pgfkeysvalueof{/pgf/outer ysep}}%  
%     \ifdim\pgf@xb<\pgf@yb%
%       \advance\pgfutil@tempdima by-\pgf@yb%
%     \else%
%       \advance\pgfutil@tempdima by-\pgf@xb%
%     \fi%
%     \pgfpathmoveto{\pgfpointadd{\pgfqpoint{\pgf@xc}{\pgf@yc}}{\pgfqpoint{-0.707107\pgfutil@tempdima}{0.707107\pgfutil@tempdima}}}
%     \pgfpathlineto{\pgfpointadd{\pgfqpoint{\pgf@xc}{\pgf@yc}}{\pgfqpoint{0.707107\pgfutil@tempdima}{-0.707107\pgfutil@tempdima}}}
%     \pgfpathmoveto{\pgfpointadd{\pgfqpoint{\pgf@xc}{\pgf@yc}}{\pgfqpoint{-0.707107\pgfutil@tempdima}{-0.707107\pgfutil@tempdima}}}
%     \pgfpathlineto{\pgfpointadd{\pgfqpoint{\pgf@xc}{\pgf@yc}}{\pgfqpoint{0.707107\pgfutil@tempdima}{0.707107\pgfutil@tempdima}}}
%   }
% }

% \tikzstyle{exrouter} = [border,inner sep=1.5mm,
%      reodist,draw=black,fill=white,circle cross]
% \tikzstyle{smallexrouter} = [border,inner sep=1.0mm,
%      reodist,draw=black,fill=white,circle cross]

\tikzstyle{exrouter}=[border,draw=black,circle,path picture={
  \draw[black](path picture bounding box.south east) -- (path picture bounding box.north west) (path picture bounding box.south west) -- (path picture bounding box.north east);}]


%--------- Setting headers ---------------
\usepackage{fancyhdr}
\usepackage{lastpage}
% \pagestyle{plain}
\pagestyle{fancy}

\newcommand{\setHeader}[2]{
  \newcommand{\myHeader}{
    \thispagestyle{empty}
    \begin{center}
      \bfseries
      {\Large #1}\\[2mm]
      \LOGO DCC-FCUP, University of Porto\\
      Jos\'{e} Proen\c{c}a\\[2mm]
      #2
    \end{center}

    % \noindent
    % Number: \rule{20mm}{0.5pt}
    % Name: \rule{116.5mm}{0.5pt}

    % \noindent
    % \hrulefill\\[0mm]
    % {\footnotesize
    % \textbf{Recall:}\\
    % \begin{minipage}{0.67\textwidth}
    %   $f(n) = \mathcal{O}(g(n))$
    %     if there exist $n0,c>0$ such that $\forall n\geq n0: f(n) \leq c \times g(n)$
    %   \\[1mm]
    %   $f(n) = \Omega(g(n))$
    %     if there exist $n0,c>0$ such that $\forall n\geq n0: f(n) \geq c \times g(n)$
    %   \\[1mm]
    %   $f(n) = \Theta(g(n))$
    %     if $f(n) = \mathcal{O}(g(n))$ and $f(n) = \Omega(g(n))$
    %   \\[1mm]
    %   Master Theorem: if $T(n)=aT(n/b) + cn^k$ then $T(N)$ is:
    %   \\\hspace*{3mm}$[\Theta(n^k)$ if $a < b^k$]
    %   ~~~[$\Theta(n^k\log n)$ if $a=b^k$]
    %   ~~~[$\Theta(n^{\log_b a}$) if $a > b^k$]
    % \end{minipage}
    % \begin{minipage}{0.31\textwidth}
    % % \begin{align*}
    %     $\sum_{i=1}^n i ~=~ \frac{n(n+1)}{2}$
    %     \\[3mm]
    %     ${\sum_{i=0}^n x^i} ~=~
    %           {\frac{x^{n+1}-1}{x-1}}$
    %     \\[3mm]
    %     ${\sum_{i=1}^n i \times x^{i-1}} ~=~ 
    %           {\frac{n\times x^{n+1}-(n+1)\times x^n +1}{(x-1)^2}}$
    % % \end{align*}
    % \end{minipage}
    % \\[2mm]
    % Answer below each question. If you need more space, use the empty space at the end of the exam.
    % \\[-1mm]If needed ask for a new paper.
    % \\[-6mm]
    % }

    \noindent
    \hrulefill
  }

  \newcommand{\newExamPage}{
    \clearpage
    \thispagestyle{empty}
    % ~
    % \newpage
    \begin{center}
      \bfseries
      {\Large #1} (continuation -- \thepage/\pageref{LastPage})\\[2mm]
      DCC-FCUP, University of Porto,
      Jos\'{e} Proen\c{c}a,
      #2
    \end{center}

    % \noindent
    % % \hrulefill
    % % \\[2mm]
    % Number: \rule{20mm}{0.5pt}
    % Name: \rule{116.5mm}{0.5pt}
  }

  \renewcommand{\headrulewidth}{.4mm} % header line width
  \fancyhf{}
  % \fancyhfoffset[L]{1cm} % left extra length
  % \fancyhfoffset[R]{1cm} % right extra length
  \rhead{\thepage/\pageref{LastPage}}
  \lhead{\bfseries #1}
  \rfoot{}
}
%-----------------------------------------


%%% Misc

\def\pv#1#2{\langle #1 \rangle #2}
\def\nc#1#2{[#1]#2}

\newcommand{\dkb}[1]{{\textcolor{blue}{#1}}}
\def\tran#1{\stackrel{#1}{\longrightarrow}}
\def\imp{\mathbin{\Rightarrow}}
\def\comp{\mathbin{\boldsymbol{\cdot}}}


% \newcommand{\note}[1]{
%   \noindent\emph{\color{black!50}\textbf{Note:} #1}
% }

\newcommand{\note}[1]{\[\textcolor{black!60}{
  \!\left[~\begin{minipage}{0.964\textwidth}{\small\it \textbf{Note:} #1}\end{minipage}~\right]
}\]}
%-------------- template --------------------------------------------------
 %!TEX root=../../1-intro.tex
 \usetheme{metropolis}
\usepackage{appendixnumberbeamer}

%------ Setting lecture info ----------------------------------------------
\newcounter{lectureID}
\stepcounter{lectureID}
\newcommand{\getLecture}{\arabic{lectureID}\xspace}
\newcommand{\setLectureBasic}[1]{
  \title{
    #1
    }
  \author{Nelma Moreira ~~\&~~ \textbf{Jos\'{e} Proen\c{c}a}}
  \institute{CISTER -- U.Porto, Porto, Portugal
            \hfill 
            \begin{tabular}{r@{}}
            \url{https://fm-dcc.github.io/pc2324}
            \end{tabular}
            }
  \date{Concurrent programming (CC3040) 2023/2024}
  % logos of institutions
  \titlegraphic{
    \begin{textblock*}{5cm}(2.0cm,7.00cm)
       \includegraphics[scale=0.14]{src/img/logos/fcup}\hspace*{.85cm}~%
    \end{textblock*}
    \begin{textblock*}{5cm}(6.0cm,7.25cm)
      \includegraphics[scale=0.43]{src/img/logos/cmup}
    \end{textblock*}
    \begin{textblock*}{5cm}(10.4cm,7.45cm)
      \includegraphics[scale=0.16]{src/img/logos/cister}
    \end{textblock*}
  }  
  \frame[plain]{\titlepage}
}
\newcommand{\setLecture}[2]{\setcounter{lectureID}{#1}\setLectureBasic{#1. #2}}

%------ Counters for exercises ----------------------------------------------
\newcounter{cExercise}
\newcommand{\exercise}{\stepcounter{cExercise}Ex.\,\arabic{lectureID}.\arabic{cExercise}:\xspace}
\newcommand{\exerciseBack}{\addtocounter{cExercise}{-1}}
\newcommand{\exerciseAdd}{\stepcounter{cExercise}}
\newcommand{\doExercise}[3][0mm]{\begin{exampleblock}{\exercise #2}\wrap{\rule{0pt}{#1}}#3\end{exampleblock}}
\newcommand{\doSimpleExercise}[2][0mm]{\begin{exampleblock}{}\wrap{\rule{0pt}{#1}}\structure{\textbf{\exercise} #2}\end{exampleblock}}

% Slide
\newenvironment{slide}[1]{\begin{frame}\frametitle{#1}}{\end{frame}}



% Base colors (from metropolis theme)
\definecolor{metDarkBrown}{HTML}{604c38}
\definecolor{metDarkTeal}{HTML}{23373b}
\definecolor{metLightBrown}{HTML}{EB811B}
\definecolor{metLightGreen}{HTML}{14B03D}

 

\metroset{numbering=fraction,progressbar=frametitle}

% \setbeamercolor*{structure}{fg=blue!80!black}
\setbeamercolor*{structure}{fg=metLightGreen}

% % \definecolor{MainColour}{rgb}{0., 0.25, 0.8}
% \colorlet{MainColour}{blue!50!black}
% \colorlet{BgColour}{blue!10}
% \colorlet{BarColour}{blue!50!black}

% %\usetheme{CambridgeUS}%{Copenhagen}%{Frankfurt}%{Singapore}%{CambridgeUS}
% \usecolortheme[named=MainColour]{structure} 
% \useoutertheme[subsection=false]{miniframes}
% \useinnertheme{circles}
% %\useinnertheme[shadow=false]{rounded}
% \setbeamertemplate{blocks}[rounded][shadow=false]

% \setbeamercovered{transparent} 
% \setbeamertemplate{navigation symbols}{} %Remove navigation bar
% \setbeamertemplate{footline}[frame number] % add page number
% \setbeamercolor{postit}{fg=MainColour,bg=BgColour}
% \setbeamercolor{structure}{bg=black!10}
% %\setbeamercolor{palette primary}{use=structure,fg=red,bg=green}
% %\setbeamercolor{palette secondary}{use=structure,fg=red!75!black,bg=green}
% \setbeamercolor{palette tertiary}{use=structure,bg=BarColour,fg=white}
% %\setbeamercolor{palette quaternary}{fg=black,bg=green}
% %\setbeamercolor{normal text}{fg=black,bg=white}
% %\setbeamercolor{block title alerted}{fg=red,bg=green}
% %\setbeamercolor{block title example}{bg=black!10,fg=green}
\setbeamercolor{block body}{bg=black!5}

% \setbeamercolor{block title alerted}{bg=red!25}
% \setbeamercolor{block body alerted}{bg=red!10}

% \setbeamercolor{block title example}{bg={rgb:green,2;black,1;white,5}}
\setbeamercolor{block body example}{bg={rgb:green,2;black,1;white,20}}
\setbeamercolor{block body alerted}{bg={metLightBrown!25}}

% \setbeamertemplate{itemize item}{\color{black!10}$\blacksquare$}
\setbeamercolor{itemize item}{fg=metDarkTeal}
\setbeamercolor{itemize subitem}{fg=metDarkTeal}

\setbeamercolor{graybc}{fg=black,bg=black!10}
\newcommand{\myblock}[1]{\begin{beamercolorbox}[dp=1ex,center,rounded=true]%
  {graybc} {\large \textbf{#1}} \end{beamercolorbox}}%

%%%%%%%%%

\definecolor{barcolor}{rgb}{.65,.79,.92} % FCUP color


% Configuring the foot line
\setbeamercolor{author in head/foot}{fg=metDarkTeal, bg=barcolor}%
\setbeamercolor{date in head/foot}{fg=barcolor!75!black}%
\setbeamertemplate{footline}
{
  \leavevmode%
  \hbox{%
  \begin{beamercolorbox}[wd=.4\paperwidth,ht=2.25ex,dp=1ex,center]{author in head/foot}%
    \usebeamerfont{author in head/foot}\insertshortauthor
  \end{beamercolorbox}%
  \begin{beamercolorbox}[wd=.5\paperwidth,ht=2.25ex,dp=1ex,center]{title in head/foot}%
    \usebeamerfont{title in head/foot}\insertsection
  \end{beamercolorbox}%
  \begin{beamercolorbox}[wd=.1\paperwidth,ht=2.25ex,dp=1ex,right]{date in head/foot}%
    \insertframenumber{} / \inserttotalframenumber\hspace*{2ex} 
  \end{beamercolorbox}}%
  \vskip0pt%
}
% No configuration symbols
\setbeamertemplate{navigation symbols}{}


%%%%%%% Custom bar above

\setbeamercolor{frametitle}{bg=barcolor,fg=metDarkTeal}
\setbeamercolor{progress bar}{fg=barcolor!75!black}

%%% Custom bar with a LOGO

\makeatletter
\setbeamertemplate{frametitle}{%
  \nointerlineskip%
  \begin{beamercolorbox}[%
      wd=\paperwidth,%
      sep=0pt,%
      leftskip=\metropolis@frametitle@padding,%
      rightskip=\metropolis@frametitle@padding,%
    ]{frametitle}%
  \metropolis@frametitlestrut@start%
  \insertframetitle%
  \nolinebreak%
  \metropolis@frametitlestrut@end%
  \hfill
  \raisebox{-1.3ex}{\includegraphics[height=4ex,keepaspectratio]{src/img/logos/fcup-bar}}
  \end{beamercolorbox}%
}
\makeatother





%----------------------------------------------------------------------------

\begin{document}

\setLecture{4}{Basic building blocks of concurrency}


\section{Overview}

\begin{frame}{We are here}

  \vspace*{-2mm}

  \begin{block}{Blocks of sequential code running concurrently and sharing memory:}
    
  \begin{itemize}
    \item What is Scala?
    \item Concurrency in Java and its memory model
    \alert{\item Basic concurrency blocks and libraries}
    % \item Futures and promises
    \item \textcolor{gray}{\emph{Futures and Promises}}
    \item \textcolor{gray}{\emph{Data-Parallel Collections}}
    \item \textcolor{gray}{\emph{Reactive Programming (Concurrently)}}
    \item \textcolor{gray}{\emph{Software Transactional Memory}}
    \item Actor model
  \end{itemize}
  \end{block}
\end{frame}



% \fromBookW[scale=0.7]{32}{98mm}{39mm}
% \fromBook[scale=0.65]{32}{43mm}{98mm}{43mm}{39mm}

% \begin{frame}[fragile]\frametitle{Current thread}
% ~\\[-8mm]
% \begin{columns}
% \begin{column}{0.49\textwidth}
% \begin{lstlisting}
% ...
% \end{lstlisting}
% \end{column}
% \begin{column}{0.49\textwidth}
% ...
% \end{column}
% \end{columns}
% \end{frame}


\begin{frame}[t]\frametitle{What we will see}

  \begin{itemize}
    \item Tread pools: Executor and ExecutionContext
    \item Non-blocking synchronisation -- compare-and-set (CAS)
    \item Lazy (concurrent) values
    \item Concurrent collections
    \item Running OS processes
  \end{itemize}


\end{frame}

\section{Existing thread pools in Scala}


\begin{frame}[fragile]\frametitle{Executor interface}
~\\[-8mm]
\begin{columns}
\begin{column}{0.54\textwidth}
\begin{lstlisting}[emph={executor},mathescape]
 Executor executor = $\textit{anExecutor}$;
 executor.execute(new RunnableTask1());
 executor.execute(new RunnableTask2());
 ...
\end{lstlisting}
\begin{lstlisting}[emph={executor}]
import scala.concurrent._
import java.util.concurrent.ForkJoinPool

object ExecutorsCreate extends App {
  val executor = new ForkJoinPool
  executor.execute(new Runnable {
    def run() = log("This task is run asynchronously.")
  })
  Thread.sleep(500) // not needed with fork:=false in SBT
}
\end{lstlisting}
\end{column}
\begin{column}{0.49\textwidth}
\begin{itemize}
  \item \alert{Executor:} can start a new thread, an existing one, or the current one
  \item Abstracts from the management of threads
  \item \structure{ExecutorService:} API that extends Executor with \code{shutdown}
    \begin{itemize}
      \item \code{executor.shutdown} $\to$ executes all tasks and then stops working threads
      \item \code{executor.awaitTermination(...)} $\to$ force termination if, after a given time, the tasks are not completed
    \end{itemize}
\end{itemize}
\end{column}
\end{columns}
\end{frame}


\begin{frame}[fragile]\frametitle{Scala's \texttt{ExecutionContext}}
~\\[-6mm]
\begin{columns}
\begin{column}{0.74\textwidth}
\begin{lstlisting}[emph={execute,sleep,log,global}]
import scala.concurrent._
object ExecutionContextGlobal extends App {
  val ectx = ExecutionContext.global
  ectx.execute(new Runnable {
    def run() = log("Running on the execution context.")
  })
  Thread.sleep(500)
}
\end{lstlisting}
\begin{lstlisting}[emph={execute,sleep,log,global,fromExecutorService}]
object ExecutionContextCreate extends App {
  val pool = new forkjoin.ForkJoinPool(2)
  val ectx = ExecutionContext.fromExecutorService(pool)
  ectx.execute(new Runnable {
    def run() = log("Running on the execution context again.")
  })
  Thread.sleep(500)
}
\end{lstlisting}
\end{column}
\begin{column}{0.36\textwidth}
\begin{itemize}
  \item \code{scala.concurrent}: has \alert{ExecutionContext}
  \item Similar to \structure{Executor} but more Scala specific
  \item often used as implicit parameter
  \item \alert{\code{global}}: default execution context (internally uses a \code{ForkJoinPool})
  \item \code{fromExecutorService}: creates \structure{ExecutionContext} from \structure{ExecutorService}
\end{itemize}
\end{column}
\end{columns}
\end{frame}


\begin{frame}[fragile]\frametitle{Simplifying the execution}
~\\[-8mm]
\begin{columns}[t]
\begin{column}{0.45\textwidth}
Similar to \structure{threads}:

\begin{lstlisting}[emph={thread,run,start}]
def thread(body: =>Unit): Thread = {
  val t = new Thread {
    override def run() = body
  }
  t.start()
  t
}
\end{lstlisting}
\end{column}
\begin{column}{0.56\textwidth}
We now define \alert{\code{execute}}

\begin{lstlisting}[emph={execute,sleep,log,global}]
def execute(body: =>Unit) = ExecutionContext.global.execute(
     new Runnable { def run() = body }
)
// For example:
object ExecutionContextSleep extends App {
  for (i<- 0 until 32) execute {
    Thread.sleep(2000)
    log(s"Task \§i completed.")
  }
  Thread.sleep(10000)
}
\end{lstlisting}
\end{column}
\end{columns}
\end{frame}


\begin{frame}[fragile]\frametitle{Avoid blocking indefinitely}
~\\[-8mm]
\begin{columns}
\begin{column}{0.45\textwidth}
\begin{lstlisting}[emph={execute,sleep,log,global}]
object ExecutionContextSleep
       extends App {
  for (i<- 0 until 32) execute {
    Thread.sleep(2000)
    log(s"Task \§i completed.")
  }
  Thread.sleep(10000)
}
\end{lstlisting}
\end{column}
\begin{column}{0.57\textwidth}
\begin{itemize}
  \item \structure{Expected:} all executions terminate after 2s
  \item \alert{Result:} only some execute after 2s %threads block forever (?!)
  \pause\\[4mm]
  \item Using quad-core CPU with hyper threading
  \item \code{global} has 8 threads in the thread pool
  \pause\\[4mm]
  \item executes tasks in batches of 8
  \item after 2s, 8 tasks print "completed"
  \item after 2s more, 8 more print "completed"
  \item \structure{sleep}: all enter a \structure{timed waiting state}
  \pause\\[4mm]
  \item if T1 \structure{waits} for T10 to \structure{notify}: \alert{blocks indefinitely}
\end{itemize}
\end{column}
\end{columns}
\end{frame}



\section{Lock-free programming}

\begin{frame}[fragile]\frametitle{Avoiding \texttt{syncrhonized} with atomic variables}
~\\[-8mm]
\begin{columns}
\begin{column}{0.57\textwidth}
\begin{itemize}
  \item \alert{atomic variable}: memory location that supports \alert{complex linearizable operations}
  \item ... i.e., \structure{appears to occur atomically}
  \item \code{write} of a volatile operation:
      \\ \structure{simple} linearizable operation
  \item at least two reads and/or writes:
      \\ \structure{complex} linearizable operation
  \pause
  \item \alert{java.util.concurrent.atomic} supports some complex ones:
    \begin{itemize}
       \item AtomicBoolean
       \item AtomicInteger
       \item AtomicLong
       \item AtomicReference  
    \end{itemize}
\end{itemize}
\end{column}
\begin{column}{0.45\textwidth}
~\\
\textbf{Variation of Example 1 (\code{getUniqueId})}
\begin{lstlisting}[emph={execute,sleep,log,global,AtomicLong,incrementAndGet}]
import java.util.concurrent.atomic._

object AtomicUid extends App {
  private val uid =
    new AtomicLong(0L)

  def getUniqueId(): Long =
    uid.incrementAndGet()
  execute {
    log(s"Uid asynchronously: \§{getUniqueId()}")
  }
  log(s"Got a unique id: \§{getUniqueId()}")
}
\end{lstlisting}
\end{column}
\end{columns}
\end{frame}


\begin{frame}[fragile]\frametitle{Compare-And-Set (CAS) -- the \alert{$\heartsuit$} of complex linearizable operations}
~\\[-8mm]
\begin{columns}
\begin{column}{0.48\textwidth}
\begin{itemize}
  \item CAS can be used to implement others:
    \begin{itemize}
      \item getAndSet
      \item decrementAndGet
      \item addAndGet
    \end{itemize}
  \item available in all atomic variables
  \item including AtomicReference[T]
\end{itemize}
\end{column}
\begin{column}{0.54\textwidth}
~\\
\textbf{Long-CAS conceptually equivalent to:}
\begin{lstlisting}[emph={execute,sleep,log,Long,synchronized}]
def compareAndSet(ov: Long, nv: Long):
          Boolean = this.synchronized {
    if (this.get != ov) false else {
      this.set(nv)
      true
  } }
\end{lstlisting}
\textbf{Ref-CAS conceptually equivalent to:}
\begin{lstlisting}[emph={execute,sleep,log,T,synchronized}]
def compareAndSet(ov: T, nv: T):
          Boolean = this.synchronized {
  if (!(this.get eq ov)) false else {
    this.set(nv)
    true
} }
\end{lstlisting}
\end{column}
\end{columns}
\end{frame}


\begin{frame}[fragile]\frametitle{Using CAS}
~\\[-8mm]
\begin{columns}
\begin{column}{0.48\textwidth}
\begin{itemize}
  \item Back to Example~1 (getUniqueId)
  \item Need to keep-on-trying
  \item Looks like busy-waiting, but it is much better
  \item Here: using (cheap) recursion instead of a loop
\end{itemize}
\end{column}
\begin{column}{0.54\textwidth}
~\\
\begin{lstlisting}[emph={execute,sleep,log,compareAndSet,synchronized}]
@tailrec def getUniqueId(): Long = {
  val oldUid = uid.get
  val newUid = oldUid + 1
  if (uid.compareAndSet(oldUid, newUid)) newUid
  else getUniqueId()
}\end{lstlisting}
\end{column}
\end{columns}
\end{frame}


\begin{frame}[fragile]\frametitle{Lock-free programming -- really?}
~\\[-8mm]
\begin{columns}
\begin{column}{0.48\textwidth}
\begin{itemize}
  \item \alert{Lock-free programs}: without locks (with \code{synchronized})
  \item Achieved using \alert{atomic variables} (and some re-trying)
  \item No locks, no deadlocks...
  \pause
  \item (almost):
    \begin{itemize}
      \item lock-free $\Rightarrow$ use atomic variables (for atomicity)
      \item use atomic variables $\cancel\Rightarrow$ lock-free
    \end{itemize}
  \pause
\end{itemize}
\end{column}
\begin{column}{0.54\textwidth}
~\\
\begin{lstlisting}[emph={execute,sleep,log,compareAndSet,mySynchronized,AtomicBoolean}]
object AtomicLock extends App {
  private val lock = new AtomicBoolean(false)
  def mySynchronized(body: =>Unit): Unit = {
    while (!lock.compareAndSet(false, true)) {}
    try body finally lock.set(false)
  }
  var count = 0
  for (i<- 0 until 10) execute { mySynchronized { count += 1 } }
  Thread.sleep(1000)
  log(s"Count is: \§count")
}
\end{lstlisting}
\end{column}
\end{columns}
\end{frame}  


\begin{frame}\frametitle{Lock-freedom definition}

\begin{alertblock}{Lock-freedom}
  Given a set of threads and an operation \alert{OP}.
  \\
  \alert{OP} is \structure{lock-free} if at least one thread always completes \alert{OP} after a finite number of steps, regardless of the speed at which different threads progress.
\end{alertblock}
\end{frame}


\begin{frame}\frametitle{One more example: Concurrent filesystem}
    \begin{itemize}
      \item \textbf{Example 1:} getUniqueId()
      \item \textbf{Example 2:} Logging Bank Transfers
      \item \textbf{Example 3:} Thread pool
      \item \textbf{Example 4:} Batman
      \item \alert{\textbf{Example 5:} Concurrent filesystem}
    \end{itemize}
\end{frame}


\begin{frame}\frametitle{Concurrent filesystem}
  \splittwo{0.35}{0.68}{
    \begin{block}{Filesystem API}
      \alert{T1} is creating \structure{F}:
      \\\alert{T2} cannot copy or delete \structure{F}
      \\[4mm]\alert{T1}\,\&\,\alert{T2} are copying \structure{F}:
      \\\alert{T3} cannot delete \structure{F}
      \\[4mm]\alert{T1} is deleting \structure{F}:
      \\\alert{T2} cannot copy nor delete \structure{F}
    \end{block}
  }{
    \fromBook[scale=0.8]{79}{40mm}{121mm}{40mm}{67mm}
  }
\end{frame}


\begin{frame}[fragile]\frametitle{Concurrent filesystem -- Scala data types}
~\\[-8mm]
\begin{columns}
\begin{column}{0.35\textwidth}
    \begin{block}{Filesystem API}
      \alert{T1} is creating \structure{F}:
      \\\alert{T2} cannot copy or delete \structure{F}
      \\[4mm]\alert{T1}\,\&\,\alert{T2} are copying \structure{F}:
      \\\alert{T3} cannot delete \structure{F}
      \\[4mm]\alert{T1} is deleting \structure{F}:
      \\\alert{T2} cannot copy nor delete \structure{F}
    \end{block}
\end{column}
\begin{column}{0.68\textwidth}
\begin{lstlisting}[emph={execute,sleep,log,compareAndSet,State,AtomicReference}]
class Entry(val isDir: Boolean) {
  val state = new AtomicReference[State](new Idle)
}

sealed trait State
class Idle extends State
class Creating extends State
class Copying(val n: Int) extends State
class Deleting extends State
\end{lstlisting}
\end{column}
\end{columns}
\end{frame}


\begin{frame}[fragile]\frametitle{Deleting and Copying}
\alert{Deleting:} \textbf{\structure{\underline{prepare}}} (checks for permission) then \structure{delete} (perform delete)

\alert{Copying:} \structure{aquire} (get permission); \structure{copy} (perform action); then \structure{release} (give permission)
\end{frame}


\begin{frame}[fragile]\frametitle{Prepare for deleting}

\begin{lstlisting}[emph={execute,sleep,log,logMessage,compareAndSet,State,AtomicReference}]
@tailrec
private def prepareForDelete(entry: Entry): Boolean = {
  val  s0 = entry.state.get
  s0 match {
    case i: Idle =>
      if (entry.state.compareAndSet(s0, new Deleting)) true
      else prepareForDelete(entry)
    case c: Creating =>
      logMessage("File currently created, cannot delete."); false
    case c: Copying =>
      logMessage("File currently copied, cannot delete."); false
    case d: Deleting =>
      false
  }
}
\end{lstlisting}

\cod{logMessage}: presented later -- similar to \cod{log}, but stores the log message
\end{frame}


\begin{frame}\frametitle{Bad copy -- the ABA problem}
\alert{``A\structure{B}A'' problem}: two readings of the same value \alert{A} lead to believe that \structure{B} was never present (type of race condition)

\bigskip

Illustrated by a bad acquire-release for \cod{Copying}, using a mutable \cod{n} in:
\\~~\alert{\texttt{Copying(\structure{var} n: Int)}}
\end{frame}


\begin{frame}[fragile]\frametitle{Bad code -- acquire/release Copying}
\label{slide:badcopy}
    
\begin{lstlisting}[emph={execute,sleep,log,logMessage,compareAndSet,releaseCopy}]
def releaseCopy(e: Entry): Copying = e.state.get match {
  case c: Copying =>
    val nstate = if (c.n == 1) new Idle else new Copying(c.n - 1)
    if (e.state.compareAndSet(c, nstate)) c
    else releaseCopy(e)
}
\end{lstlisting}
\begin{lstlisting}[emph={execute,sleep,log,logMessage,compareAndSet,acquireCopy}]
def acquireCopy(e: Entry, c: Copying) = e.state.get match {
  case i: Idle =>
    c.n = 1
    if (!e.state.compareAndSet(i, c)) acquireCopy(e, c)
  case oc: Copying =>
    c.n = oc.n + 1
    if (!e.state.compareAndSet(oc, c)) acquireCopy(e, c)
}
\end{lstlisting}
Optimization: reusing previous \cod{Copying} if possible
\end{frame}


\begin{frame}\frametitle{Bad trace -- T1\&T2 release; T3\&T1 aquire -- (T2 is slow)}
    \fromBookW{82}{151mm}{33mm}
\end{frame}


\begin{frame}\frametitle{Some guidelines to avoid the ABA problem}
  \begin{itemize}
    \item use fresh objects in \cod{AtomicReference}
    \item use immutable objects in \cod{AtomicReference}
    \item avoid re-assigning the same value to an atomic variable
    \item only increment or decrement values of numeric atomic variables (if possible)
  \end{itemize}


\end{frame}


\section{Lazy values}

\begin{frame}[fragile]\frametitle{Lazy values can cause deadlocks}
~\\[-8mm]
\begin{columns}
\begin{column}{0.48\textwidth}
\begin{itemize}
  \item \alert{\textbf{lazy values:}} initialized when read for the first time
  \item these should not depend-on/modify state (non-determinism)
  \item code in \structure{singleton objects}: lazy execution
  \item under the hood: first write uses a lock -- to ensure at most a thread initialises a lazy value
  \item \structure{stack overflow} (sequential code)\\~~can become\\
        \alert{deadlock} (concurrent code)

\end{itemize}
\end{column}
\begin{column}{0.54\textwidth}
~\\
\begin{lstlisting}[emph={execute,sleep,log,compareAndSet,mySynchronized,AtomicBoolean}]
object LazyValsCreate extends App {
  var x = 5
  lazy val y = x+2
  execute {log(s"Wrk: y = $y")}
  x = 10
  log(s"Main: y = $y")
  // y = 7 or 12 in both cases
  Thread.sleep(500)
}
\end{lstlisting}
\begin{lstlisting}[emph={execute,sleep,log,compareAndSet,mySynchronized,AtomicBoolean}]
object LazyValsDeadlock extends App {
  object A { lazy val x: Int = B.y }
  object B { lazy val y: Int = A.x }
  execute { B.y }
  A.x
}
\end{lstlisting}
\end{column}
\end{columns}
\end{frame}

\section{Concurrent (mutable) collections}


\begin{frame}[fragile]\frametitle{Default mutable collections are not concurrent}


~\\[-8mm]
\begin{columns}
\begin{column}{0.48\textwidth}
\begin{itemize}
  \item Naive code: arbitrarily returns different results and exceptions
  \item Corruption of the internal state
  \item Possible fixes:
    \begin{itemize}
      \item immutable collections + atomic variables
      \\\only<2>{\structure{(does not scale)}}
      \item mutable collections + synchronized
      \\\only<2>{\structure{(assuming collections do not block; may not scale)}}
      \item dedicated libraries
      \\\only<2>{\structure{(far better performance and scalability)}}
    \end{itemize}
\end{itemize}
\end{column}
\begin{column}{0.54\textwidth}
~\\
\begin{lstlisting}[emph={execute,sleep,log,compareAndSet,asyncAdd}]
import scala.collection._
object CollectionsBad extends App {
  val buffer = mutable.ArrayBuffer[Int]()
  def asyncAdd(numbers: Seq[Int]) = execute {
    buffer ++= numbers
    log(s"buffer = $buffer")
  }
  asyncAdd(0 until 10)
  asyncAdd(10 until 20)
  Thread.sleep(500)
}
\end{lstlisting}
\end{column}
\end{columns}
\end{frame}


\begin{frame}[t]\frametitle{Some concurrent collections}
    
  \begin{itemize}
    \item Concurrent queues 
      \begin{itemize}
         \item java.util.concurrent.BlockingQueue \structure{interface}
        \item ...ArrayBlockingQueue \alert{class} (bounded)
        \item ...LinkedBlockingQueue \alert{class} (unbounded)
       \end{itemize} 
    \item Concurrent Sets and Maps
      \begin{itemize}
        \item scala.collection.concurrent.Map \structure{trait}
        \item java.util.concurrent.ConcurrentHashMap \alert{class}
      \end{itemize}
  \end{itemize}
\end{frame}

\begin{frame}[t]\frametitle{BlockingQueue API}
    \centering
    \fromBook[scale=1.2]{90}{20mm}{24mm}{59mm}{172mm}
\end{frame}

\begin{frame}[fragile]\frametitle{Back to Example 5: logging in our concurrent filesystem}
    
We will compile a queue of messages when \alert{logging} messages in our file system


\begin{lstlisting}[emph={execute,sleep,log,compareAndSet,logger,messages,logMessage}]
class FileSystem(...) {
  ...
  private val messages = new LinkedBlockingQueue[String]
  val logger = new Thread {
    setDaemon(true)
    override def run() = while (true) log(messages.take())
  }
  logger.start()
  def logMessage(msg: String): Unit = messages.offer(msg)
}
\end{lstlisting}

\begin{lstlisting}[emph={execute,sleep,log,compareAndSet,logMessage}]
...
val fileSystem = new FileSystem(".") // to be defined later
fileSystem.logMessage("Testing log!")
\end{lstlisting}
\end{frame}


\begin{frame}\frametitle{Note on iterators}

\begin{itemize}
  \item concurrentQueue.iterator
  \item can produce inconsistent results
  \item while traversing and modifying, the iterator can be updated
  \item (heavier) exceptions create a copy when producing an iterator
\end{itemize}
\end{frame}


\begin{frame}[fragile]\frametitle{Example 5: files as a Map in our FileSystem}

\begin{lstlisting}[emph={execute,sleep,log,ConcurrentHashMap,files,Entry,iterateFiles}]
import scala.collection.convert.decorateAsScala._
import java.io.File
import org.apache.commons.io.FileUtils // needs "commons-io" in build.sbt

class FileSystem(val root: String) {
  val rootDir = new File(root)
  val files: concurrent.Map[String, Entry] =
    new ConcurrentHashMap().asScala
  for (f <- FileUtils.iterateFiles(rootDir, null, false).asScala)
    files.put(f.getName, new Entry(false))

  ...
}
\end{lstlisting}
\end{frame}


\begin{frame}[fragile]\frametitle{Deleting a file}

Recall the \code{prepareForDelete(entry)}

\begin{lstlisting}[emph={execute,sleep,log,deleteFile,files,prepareForDelete,deleteQuietly}]
...
def deleteFile(filename: String): Unit = {
  files.get(filename) match {
    case None =>
      logMessage(s"Path '$filename' does not exist!")
    case Some(entry) if entry.isDir =>
      logMessage(s"Path '$filename' is a directory!")
    case Some(entry) => execute {
      if (prepareForDelete(entry))
        if (FileUtils.deleteQuietly(new File(filename)))
          files.remove(filename)
    }
  }
}
\end{lstlisting}
\end{frame}


\begin{frame}\frametitle{Some complex linearizable methods of concurrent Map}
    \centering
    \fromBookW[scale=0.8]{95}{34mm}{142mm}

    \splittwo{0.48}{0.48}{
      \begin{itemize}
        \item These use ``equals'' instead of the reference (which CAS does)
        \item Avoid \code{null} as key or valye (often used as special values)
      \end{itemize}
    }{
      \begin{itemize}
        \item Methods \code{+=}, \code{-=}, \code{put}, \code{update}, \code{get}, \code{apply}, \code{remove} are (non-complex) linearizable
      \end{itemize}
    }
    % (Using "equals" instead of the reference, as with CAS)
\end{frame}


\section*{Wrapping up our Filesystem (Example 5)}

\begin{frame}[fragile]\frametitle{Copying in our Filesystem}
    Recall our \alert{broken} \structure{aquireCopy}/\structure{releaseCopy} methods (\alert{ABA problem}) -- slide\ref{slide:badcopy}

\begin{lstlisting}[emph={execute,sleep,log,Copying,aquire,compareAndSet}]
@tailrec
private def acquire(entry: Entry): Boolean = {
  val s0 = entry.state.get
  s0 match {
    case _: Creating | _: Deleting =>
      logMessage("File inaccessible, cannot copy."); false
    case i: Idle =>
      if (entry.state.compareAndSet(s0, new Copying(1))) true
      else acquire(entry)
    case c: Copying =>
      if (entry.state.compareAndSet(s0, new Copying(c.n+1))) true
      else acquire(entry)
  }
}
\end{lstlisting}
\end{frame}


\begin{frame}[fragile]\frametitle{Copying in our Filesystem}
    Same CAS retry-approach for releasing.

\begin{lstlisting}[emph={execute,sleep,log,Copying,release,compareAndSet}]
@tailrec
private def release(entry: Entry): Unit = {
  val s0 = entry.state.get
  s0 match {
    case c: Creating =>
      if (!entry.state.compareAndSet(s0, new Idle)) release(entry)
    case c: Copying =>
      val nstate = if (c.n == 1) new Idle else new Copying(c.n-1)
      if (!entry.state.compareAndSet(s0, nstate)) release(entry)
  }
}
\end{lstlisting}
\end{frame}


\begin{frame}[fragile]\frametitle{Copying in our Filesystem}
    Finally: wrapper for copying a file.

\begin{lstlisting}[emph={execute,sleep,log,Copying,copyFile,acquire, release}]
def copyFile(src: String, dest: String): Unit = {
  files.get(src) match {
    case Some(srcEntry) if !srcEntry.isDir => execute {
      if (acquire(srcEntry)) try {
        val destEntry = new Entry(isDir = false)
        destEntry.state.set(new Creating)
        if (files.putIfAbsent(dest, destEntry) == None) try {
          FileUtils.copyFile(new File(src), new File(dest))
        } finally release(destEntry)
      } finally release(srcEntry)
    }
} }
\end{lstlisting}
\end{frame}



\section{Creating and handling processes}

\begin{frame}\frametitle{Beyond JVM}

\begin{itemize}
  \item \structure{So far:} run in a single JVM
  \item \structure{Now:} run processes outside JVM
  \item \structure{Why:}
  \pause
    \begin{itemize}
      \item Some programs do not exist in Scala/Java
      \item \pause Want to sandbox untrusted code
      \item \pause Performance (running independent code)
    \end{itemize}
  \pause
  \item Using the \alert{\code{scala.sys.process}} package
\end{itemize}
\end{frame}

\begin{frame}[fragile]\frametitle{Using processes - examples}
    
\begin{lstlisting}[emph={execute,sleep,log,destroy,!}]
import scala.sys.process._
object ProcessRun extends App {
  val command = "ls"
  val exitcode = command.! // run process (with side effects)
  log(s"command exited with status $exitcode") }
\end{lstlisting}

\begin{lstlisting}[emph={execute,sleep,log,destroy,!}]
def lineCount(filename: String): Int = {
  val output = s"wc $filename".!! // run and retreive stdout
  output.trim.split(" ").head.toInt }
\end{lstlisting}

\begin{lstlisting}[emph={execute,sleep,log,destroy}]
object ProcessAsync extends App {
  val lsProcess = "ls -R /".run() // run and returns a Process object
  Thread.sleep(1000)
  log("Timeout - killing ls!")
  lsProcess.destroy() } // kill a slow process
\end{lstlisting}

{\small\url{https://www.scala-lang.org/api/2.13.x/scala/sys/process/ProcessBuilder.html}}

\end{frame}


\begin{frame}\frametitle{Wrapping up ``concurrency blocks''} %{Summary of concurrency blocks covered}
  \splittwo{0.56}{0.44}{
    \begin{itemize}
      \item \texttt{\color{blue!75!black}executor.execute(...)}
      \item lock-free programming with atomic variables
      \item \texttt{\color{blue!75!black}av.compareAndSet(...)}
      \item ABA problem
      \item Lazy values \& ``lazy'' objects
      \item \texttt{\color{blue!75!black}java.util.concurrent.BlockingQueue}
      \item \texttt{\color{blue!75!black}scala.collection.concurrent.Map}
      \item \textcolor{gray}{\emph{weakly consistent iterators}}
      \item \textcolor{gray}{\emph{custom concurrent data structures}}
    \end{itemize}
  }{
    \begin{itemize}
      \item Filesystem example
      \item Processes outside JVM
    \end{itemize}
    \pause
    \begin{alertblock}{Next}
      \begin{itemize}
        \item \textcolor{gray}{\emph{Futures and Promises}}
        \item \textcolor{gray}{\emph{Data-Parallel Collections}}
        \item \textcolor{gray}{\emph{Reactive Programming (Concurrently)}}
        \item \textcolor{gray}{\emph{Software Transactional Memory}}
        \item \alert{Actors}
      \end{itemize}
    \end{alertblock}
  }
\end{frame}


\end{document}
