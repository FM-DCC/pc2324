\documentclass[aspectratio=169]{beamer}
\usepackage{etex} % fixes new-dimension error
\usepackage{lmodern}
\usepackage[T1]{fontenc}
\usepackage{graphicx,amsmath}
\usepackage{stmaryrd} % cf. interleave
\usepackage{booktabs}
\usepackage{amscd}
\usepackage{multicol}
\usepackage[absolute,overlay]{textpos}
\usepackage{alltt}
\usepackage{proof}
\usepackage{subcaption}
\usepackage{tikz}
\usepackage{tikz-cd}
\usepackage[new]{old-arrows}
\usepackage[all]{xy}
\usepackage{pgfplots}
\usepackage{textcomp}
\usepackage{cancel}

%------ using color ---------------------------------------------------------
%\newrgbcolor{goldenrod}{.80392 .60784 .11373}
%\newrgbcolor{darkgoldenrod}{.5451 .39608 .03137}
%\newrgbcolor{brown}{.15 .15 .15}
%\newrgbcolor{darkolivegreen}{.33333 .41961 .18431}
\definecolor{goldenrod}{rgb}{.80392,.60784,.11373}
\definecolor{darkgoldenrod}{rgb}{.5451,.39608,.03137}
\definecolor{brown}{rgb}{.15,.15,.15}
\definecolor{darkolivegreen}{rgb}{.33333,.41961,.18431}
\definecolor{darkgreen}{rgb}{0,0.6,0}
\definecolor{myGray}{gray}{0.85}
%


% \newcommand{\red}[1]{\textcolor{red!80!black}{#1}\xspace}
% \newcommand{\blue}[1]{\textcolor{blue}{#1}\xspace}
% \newcommand{\gold}[1]{\textcolor{darkgoldenrod}{#1}\xspace}
\newcommand{\grey}[1]{\textcolor{myGray}{#1}\xspace}
\def\laplace#1#2{*\txt{\mbox{ \fcolorbox{black}{myGray}{$\begin{array}{c}\mbox{#1}\\\\#2\\\\\end{array}$} }}}
\newenvironment{bluein}{\blue}{\black\hskip -2.5pt}

%------ contexts  ---------------------------------------------------------

\newtheorem{defi}{Defini\cao}[section]
\newtheorem{defi*}{Defini\cao}
\newtheorem{lema*}{Lema}
\newenvironment{lsbcom}
      {\footnotesize  \hrule ~\\ ~\\ {\bf \sc Nota:} }
      {\hrule  ~\\ ~\\  \normalsize}
      
\newenvironment{lsbcomi}
      {\footnotesize  \hrule ~\\ ~\\ {\bf \sc Note:} }
      {\hrule  ~\\ ~\\  \normalsize}
      
\newenvironment{demo}%
     {\vspace{-5mm}\noindent {\bf Prova:}}%
    {\par \nopagebreak  \noindent \fimdemo \vspace{3mm} }
\newenvironment{demoi}%
     {\vspace{-5mm}\noindent {\bf Proof:}}%
    {\par \nopagebreak  \noindent \fimdemo \vspace{3mm} }


% \newcommand{\N}{\ensuremath{\mathbb{N}}\xspace}
% \newcommand{\Z}{\ensuremath{\mathbb{Z}}\xspace}
% \newcommand{\R}{\ensuremath{\mathbb{R}}\xspace}

\newcommand{\cR}{\ensuremath{\mathcal{R}}\xspace}

% \newcommand{\set}[1]{\left\{ #1 \right\}} % {a,b,...z}
\def\bang{{!}}
\def\deff{\stackrel{\rm def}{=}}          % Function definition symbol
%\def\deff{\, :=\, }          % Function definition symbol


\def\always{\boxempty}
\def\eventual{\Diamond}
\def\nexts{\bigcirc}
\def\until{\mathbin{\mathcal U}}




%------ CCS          -----------------------------------------%
\def\ppe{\mathbin{\vartriangleright}}
\def\kcomp{\mathbin{\boldsymbol{\bullet}}}
%\def\qcomp{\mathbin{\boldsymbol{;}}}
\def\ssp{\textsc{skip}}
\def\fim{\dagger}
%\def\ppe{\gg}
\def\ff{\ensuremath{\mi{f\!f}}\xspace}
\def\tt{\ensuremath{\mi{t\!t}}\xspace}
\def\cnil{\mathbf{0}}
\def\cpf#1#2{#1 . #2}                           % a.P
\def\cou#1#2{#1 \mathbin{+} #2}                 % P + Q
%\def\crt#1#2{\mathbin{#1 \setminus_{#2}}}       % P \ A
%\def\crtt#1#2{\mathbin{#1 \setminus\!\setminus_{#2}}}       % P \ A
%\def\crt#1#2{\mathsf{new}\, #2\;  #1}       % P \ A
\def\crt#1#2{#1 \backslash #2}       % P \ A
%\def\crn#1#2{\{#2\}\, #1}                  % P[f]

\def\crn#1#2{\mathbin{#1[#2]}}                  % P[f]
\def\couit#1#2{\Sigma_{#1}#2}                  %  + i=1,n
\def\cpar#1#2{#1 \mid #2}                       %  |
\def\ctpar#1#2{#1 \parallel #2}                       %  |
\def\cpars#1#2#3{#1 \mid_{#3} #2}               %  |S
\def\ffix#1#2{\underline{fix}~(#1\, =\, #2)}  % fix X
\def\fffix#1#2#3{\underline{fix}_{#1}~(#2\, =\, #3)}  % fix X
\def\tfix#1#2{\underline{\Tilde{fix}}~(\Tilde{#1}\, =\, \Tilde{#2})}  % fix X
%\def\ainv#1{\Bar{#1}}                   % ~ a
\def\ainv#1{\overline{#1}}                   % ~ a
\def\cif#1#2{\fuc{if}\, #1\, \fuc{then}\, #2}
\def\ccif#1#2#3{\fuc{if}\, #1\, \fuc{then}\, #2\, \fuc{else}\, #3}

\def\fres#1#2{#1 \restriction #2}                

\def\mean#1{\mathopen{[\![}#1\mathclose{]\!]}}
\def\llbracket{\mathopen{[\![}}
\def\rrbracket{\mathopen{]\!]}}

\def\cfree#1{\fuc{fn} (#1)}
\def\cbound#1{\fuc{bn} (#1)}
\def\anew#1{\overline{#1}\mathsf{new}\, } 
\def\transitatau{\rtran{\tau}}
\def\transitaa{\rtran{a}}


% 2015




%%%%%%%%%%%%%%%% NUNO reconf


\def\trans#1{\stackrel{#1}{\longrightarrow}}
\def\TS#1{\mathcal{G}(#1)}
\def\TSn#1{\mathcal{G}_{nodes}(#1)}




%--------------------
%--- by jose 2016 ---
%--------------------

\newcommand{\myblock}[1]{\begin{beamercolorbox}[dp=1ex,center,rounded=true]%
  {postit} {\large \textbf{#1}} \end{beamercolorbox}}%
\def\trans#1{\xrightarrow{#1}}  % - a - > 
\def\Trans#1{\stackrel{#1}{\Longrightarrow}} % =a=> 
\def\transtau{\xrightarrow{\tau}}  % - a - > 


%\newcommand{\evm}[1]{\langle #1 \rangle\,\fi}
%\newcommand{\alm}[1]{[#1]\,}
\newcommand{\evm}[1]{\if\relax\detokenize{#1}\relax 
  \Diamond \else\langle #1 \rangle\fi}
\newcommand{\alm}[1]{\if\relax\detokenize{#1}\relax 
  \boxempty \else[#1]\fi}

\newcommand{\myparagraph}[1]{\medskip\noindent\textbf{#1}~~}



% Listing
\lstset{
  language=Scala,
  basicstyle=\ttfamily\footnotesize, % overriding size
  commentstyle=\sffamily\color{green!60!black},
}
% \lstset{%language=Java
% %  ,basicstyle=\footnotesize
%   ,columns=fullflexible %space-fexible
%   ,keepspaces
% %  ,numberstyle=\tiny
%   ,mathescape=true
%   ,showstringspaces=false
% %  ,morekeywords={refract,global,local,on-change}
%   ,morecomment=[l]{\%}
%   ,commentstyle=\sl\sffamily\color{gray}\scriptsize
%   ,basicstyle=\ttfamily\relsize{-0.5}
%   ,keywordstyle=\bf\sffamily\color{purple}
% %  ,emphstyle=\it\sffamily\color{blue!80!black}
%   ,emphstyle=\bfseries\itshape\color{blue!80!black}
%   ,emphstyle={[2]\itshape\color{red!70!black}}%\underbar}
%   ,stringstyle=\color{darkgreen}
%   ,alsoletter={-,||,+,<>,&&,=>}
%   ,literate=*{->}{{{\color{red!70!black}$\to$}}}{1}
%              {.}{{{\color{red!70!black}.}}}{1}
%              {+}{{{\color{red!70!black}+}}}{1}
%              {*}{{{\color{red!70!black}*}}}{1}
%              {\#}{{{\color{red!70!black}\#}}}{1}
%   ,emph={act,proc,init,sort}
%   ,emph={[2]block,hide,comm,rename,allow,||,<>,sum,&&,=>}
% %  ,emphstyle={[2]\color{blue}}2
%   ,framerule=1pt
%   ,backgroundcolor=\color{black!2}
%   ,rulecolor=\color{black!30}
%   ,frame=tblr
%   ,xleftmargin=4pt
%   ,xrightmargin=4pt
%   ,captionpos=b
% %  ,belowcaptionskip=\medskipamount
%   ,aboveskip=\baselineskip
%   ,floatplacement=htb
% }
\lstdefinestyle{bash}{literate=*}
% \newcommand{\code}[1]{\lstinline[basicstyle=\ttfamily\relsize{-0.5},keywordstyle=\bf\sffamily\color{purple},columns=fullflexible,keepspaces]�#1�}
\newcommand{\code}[1]{\lstinline[columns=fullflexible,keepspaces]�#1�}
\newcommand{\mcode}[1]{\text{\code{#1}}}
\newcommand{\bash}[1]{\lstinline[basicstyle=\ttfamily\relsize{-0.5}\color{darkgreen},keywordstyle=\bf\sffamily\color{purple},columns=fullflexible,keepspaces,literate=*]�#1�}


%%%%%
% Named environments (no counters) 
\newenvironment{theorem}[2][Theorem]{\begin{trivlist}
\item[\hskip \labelsep {\bfseries #1}\hskip \labelsep {\bfseries #2.}]}{\end{trivlist}}
\newenvironment{lemma}[2][Lemma]{\begin{trivlist}
\item[\hskip \labelsep {\bfseries #1}\hskip \labelsep {\bfseries #2.}]}{\end{trivlist}}
%\newenvironment{exercise}[2][Exercise]{\begin{trivlist}
%\item[\hskip \labelsep {\bfseries #1}\hskip \labelsep {\bfseries #2.}]}{\end{trivlist}}
\newenvironment{problem}[2][Problem]{\begin{trivlist}
\item[\hskip \labelsep {\bfseries #1}\hskip \labelsep {\bfseries #2.}]}{\end{trivlist}}
\newenvironment{question}[2][Question]{\begin{trivlist}
\item[\hskip \labelsep {\bfseries #1}\hskip \labelsep {\bfseries #2.}]}{\end{trivlist}}
\newenvironment{corollary}[2][Corollary]{\begin{trivlist}
\item[\hskip \labelsep {\bfseries #1}\hskip \labelsep {\bfseries #2.}]}{\end{trivlist}}
 
% Environments with counters
\newtheoremstyle{myplain} {8mm}% (Space above)
{3mm}% (Space below)
{}% (Body font)
{}% (Indent amount)
{\bfseries\large}% (Theorem head font)
{.}% (Punctuation after theorem head)
{.5em}% (Space after theorem head)2
{}% (Theorem head spec (can be left empty, meaning �normal�))

\theoremstyle{myplain}
\newtheorem{myExercise}{Exercise}

\theoremstyle{definition} % no italics
\newtheorem{subexercise}{}[myExercise]

\newcommand{\ex}[1]{\begin{myExercise}#1\end{myExercise}}
\newcommand{\subex}[1]{\begin{subexercise}#1\end{subexercise}}

\newcommand{\mytikz}[2][]{\medskip\centerline{\begin{tikzpicture}[#1]#2\end{tikzpicture}}\medskip}



%%%%% REO %%%%
\pgfdeclarelayer{background}
\pgfdeclarelayer{threadground}
\pgfsetlayers{background,threadground,main}

\usepackage{pgfplots}

% thickness of the lines
\tikzstyle{border}   = [thick]
\tikzstyle{reodist}    = [node distance=15mm]

% nodes and I/O
\tikzstyle{reonode}  = [border,circle,inner sep=1.5pt,reodist]
\tikzstyle{mixed}    = [reonode, line width=0, %draw=black,
%                        outer color=black,inner color=black!20
                        shading=ball,ball color=black]
\tikzstyle{boundary} = [reonode,draw=black,fill=white]
\tikzstyle{point}    = [reonode,draw=black,fill=black,inner sep=0.5pt]
\tikzstyle{io}       = [border,rectangle,draw=black,fill=white,inner sep=3.25pt,node distance=0.75cm]
\tikzstyle{ioblack}  = [border,rounded corners=0,rectangle,draw=black,
    fill=black,inner sep=3.25pt,node distance=0.75cm]
\tikzstyle{comp}     = [border,rectangle,draw=black,fill=white,inner sep=3.25pt,reodist]
\tikzstyle{token} = [inner sep=0.8mm,regular polygon,regular polygon sides=5,draw=black, fill=#1]
\newcommand{\token}{\tikz \node[token=green!70!black] {};\xspace}


%% animations
\tikzstyle{animflow}=[blue,draw opacity=0.2,line width=2.7mm,line cap=round,line join=round]                   
\tikzstyle{animnf}=[postaction=decorate,line width=0,draw opacity=0,
                  decoration={markings,mark=at position #1 with 
                  {\node[regular polygon,regular polygon sides=3,rotate=30,draw=red,draw opacity=0.5,line width=1.5pt,line cap=round,line join=round, rounded corners=0.5pt,
                  transform shape,inner sep=1.3pt]{};}}]
\tikzstyle{animfflow}=[blue!20,line width=2.7mm,line cap=round,line join=round]                   


% channels
\tikzstyle{channel}=[border,>=stealth]
\tikzstyle{sync}=[channel,->]
\tikzstyle{lossy}=[channel,->,dashed]
\tikzstyle{sdrain}=[channel,>-<]
\tikzstyle{sspout}=[channel,<->]
\tikzstyle{fifo}=[channel,->,
                  postaction=decorate,
                  decoration={markings,mark=at position 0.5 with 
                  {\node[rectangle,draw=black,fill=white,rounded corners=0,
                  transform shape,minimum width=6mm]{#1};}}]
\tikzstyle{fifos}=[channel,->,
                  postaction=decorate,
                  decoration={markings,mark=at position 0.5 with 
                  {\node[rectangle,draw=black,fill=white,rounded corners=0,
                  transform shape,minimum width=4mm]{#1};}}]
\tikzstyle{fifocol}=[channel,->,
                  postaction=decorate,
                  decoration={markings,mark=at position 0.5 with 
                  {\node[rectangle,draw=black,fill=#1,rounded corners=0,
                  transform shape,minimum width=6mm]{};}}]
\tikzstyle{vare}=[channel,->,
                  postaction=decorate,
                  decoration={markings,mark=at position 0.5 with 
                  {\node[rectangle,draw=black,fill=white,rounded corners=3,
                  transform shape,minimum width=6mm]{#1};}}]
\tikzstyle{varf}=[channel,->,
                  postaction=decorate,
                  decoration={markings,mark=at position 0.5 with 
                  {\node[rectangle,draw=black,fill=black,rounded corners=3,
                  transform shape,minimum width=6mm]{#1};}}]
\tikzstyle{filter}=[channel,->,
                  decorate,rounded corners=0,
                  decoration={zigzag,segment length=1.3mm, 
                  pre length=4mm,post length=4mm,#1}]
\tikzstyle{lfilter}=[channel,->,
                  decorate,rounded corners=0,
                  decoration={zigzag,segment length=1.3mm, 
                  pre length=#1,post length=#1}]
\tikzstyle{llfilter}=[channel,->,
                  decorate,rounded corners=0,
                  decoration={zigzag,segment length=1.3mm, 
                  pre length=#1,post length=#1}]
\tikzstyle{transf}=[channel,->,
                  postaction=decorate,
                  decoration={markings,mark=at position 0.45 with 
                  {\node[isosceles triangle,draw=black,fill=white,rounded corners=0,
                  transform shape,inner sep=2pt,#1]{};}}]
\tikzstyle{clossy}=[channel,->,dashed,
                  postaction=decorate,
                  decoration={markings,mark=at position 0.5 with 
                  {\node[sloped]{!};}}]
\tikzstyle{pdrain}=[channel,>-<,
                  postaction=decorate,
                  decoration={markings,
                    mark=at position 0.5 with
                      {\draw[channel,-]
                        (-2pt,-3pt) -- (-2pt,3pt) [transform shape]
                        (2pt,-3pt) -- (2pt,3pt) [transform shape];},
                    mark=at position 0.25 with {\node[sloped]{!};}}]

\tikzstyle{adrain}=[channel,>-<,
    postaction=decorate,
    decoration={markings,mark=at position 0.5 with
    {\draw[channel,-] (-2pt,-3pt) -- (-2pt,3pt) [transform shape]
                      (2pt,-3pt) -- (2pt,3pt) [transform shape];}}]
\tikzstyle{aspout}=[channel,<->,
    postaction=decorate,
    decoration={markings,mark=at position 0.5 with
    {\draw[channel,-] (-2pt,-3pt) -- (-2pt,3pt) [transform shape]
                      (2pt,-3pt) -- (2pt,3pt) [transform shape];}}]


% connector box
\tikzstyle{connector}=[fill=black!10,rounded corners]

\newcommand{\reoconnector}[2][]{
  \begin{tikzpicture}[line join=round,#1]
  #2
  \end{tikzpicture}
}

%%% LOTS OF WORK for an exclusive router.

% \pgfdeclareshape{circle cross}
% {
%   \inheritsavedanchors[from=circle] % this is nearly a circle
%   \inheritanchorborder[from=circle]
%   \inheritanchor[from=circle]{north}
%   \inheritanchor[from=circle]{north west}
%   \inheritanchor[from=circle]{north east}
%   \inheritanchor[from=circle]{center}
%   \inheritanchor[from=circle]{west}
%   \inheritanchor[from=circle]{east}
%   \inheritanchor[from=circle]{mid}
%   \inheritanchor[from=circle]{mid west}
%   \inheritanchor[from=circle]{mid east}
%   \inheritanchor[from=circle]{base}
%   \inheritanchor[from=circle]{base west}
%   \inheritanchor[from=circle]{base east}
%   \inheritanchor[from=circle]{south}
%   \inheritanchor[from=circle]{south west}
%   \inheritanchor[from=circle]{south east}
%   \inheritbackgroundpath[from=circle]
%   \foregroundpath{
%     \centerpoint%
%     \pgf@xc=\pgf@x%
%     \pgf@yc=\pgf@y%
%     \pgfutil@tempdima=\radius%
%     \pgfmathsetlength{\pgf@xb}{\pgfkeysvalueof{/pgf/outer xsep}}%  
%     \pgfmathsetlength{\pgf@yb}{\pgfkeysvalueof{/pgf/outer ysep}}%  
%     \ifdim\pgf@xb<\pgf@yb%
%       \advance\pgfutil@tempdima by-\pgf@yb%
%     \else%
%       \advance\pgfutil@tempdima by-\pgf@xb%
%     \fi%
%     \pgfpathmoveto{\pgfpointadd{\pgfqpoint{\pgf@xc}{\pgf@yc}}{\pgfqpoint{-0.707107\pgfutil@tempdima}{0.707107\pgfutil@tempdima}}}
%     \pgfpathlineto{\pgfpointadd{\pgfqpoint{\pgf@xc}{\pgf@yc}}{\pgfqpoint{0.707107\pgfutil@tempdima}{-0.707107\pgfutil@tempdima}}}
%     \pgfpathmoveto{\pgfpointadd{\pgfqpoint{\pgf@xc}{\pgf@yc}}{\pgfqpoint{-0.707107\pgfutil@tempdima}{-0.707107\pgfutil@tempdima}}}
%     \pgfpathlineto{\pgfpointadd{\pgfqpoint{\pgf@xc}{\pgf@yc}}{\pgfqpoint{0.707107\pgfutil@tempdima}{0.707107\pgfutil@tempdima}}}
%   }
% }

% \tikzstyle{exrouter} = [border,inner sep=1.5mm,
%      reodist,draw=black,fill=white,circle cross]
% \tikzstyle{smallexrouter} = [border,inner sep=1.0mm,
%      reodist,draw=black,fill=white,circle cross]

\tikzstyle{exrouter}=[border,draw=black,circle,path picture={
  \draw[black](path picture bounding box.south east) -- (path picture bounding box.north west) (path picture bounding box.south west) -- (path picture bounding box.north east);}]


%--------- Setting headers ---------------
\usepackage{fancyhdr}
\usepackage{lastpage}
% \pagestyle{plain}
\pagestyle{fancy}

\newcommand{\setHeader}[2]{
  \newcommand{\myHeader}{
    \thispagestyle{empty}
    \begin{center}
      \bfseries
      {\Large #1}\\[2mm]
      \LOGO DCC-FCUP, University of Porto\\
      Jos\'{e} Proen\c{c}a\\[2mm]
      #2
    \end{center}

    % \noindent
    % Number: \rule{20mm}{0.5pt}
    % Name: \rule{116.5mm}{0.5pt}

    % \noindent
    % \hrulefill\\[0mm]
    % {\footnotesize
    % \textbf{Recall:}\\
    % \begin{minipage}{0.67\textwidth}
    %   $f(n) = \mathcal{O}(g(n))$
    %     if there exist $n0,c>0$ such that $\forall n\geq n0: f(n) \leq c \times g(n)$
    %   \\[1mm]
    %   $f(n) = \Omega(g(n))$
    %     if there exist $n0,c>0$ such that $\forall n\geq n0: f(n) \geq c \times g(n)$
    %   \\[1mm]
    %   $f(n) = \Theta(g(n))$
    %     if $f(n) = \mathcal{O}(g(n))$ and $f(n) = \Omega(g(n))$
    %   \\[1mm]
    %   Master Theorem: if $T(n)=aT(n/b) + cn^k$ then $T(N)$ is:
    %   \\\hspace*{3mm}$[\Theta(n^k)$ if $a < b^k$]
    %   ~~~[$\Theta(n^k\log n)$ if $a=b^k$]
    %   ~~~[$\Theta(n^{\log_b a}$) if $a > b^k$]
    % \end{minipage}
    % \begin{minipage}{0.31\textwidth}
    % % \begin{align*}
    %     $\sum_{i=1}^n i ~=~ \frac{n(n+1)}{2}$
    %     \\[3mm]
    %     ${\sum_{i=0}^n x^i} ~=~
    %           {\frac{x^{n+1}-1}{x-1}}$
    %     \\[3mm]
    %     ${\sum_{i=1}^n i \times x^{i-1}} ~=~ 
    %           {\frac{n\times x^{n+1}-(n+1)\times x^n +1}{(x-1)^2}}$
    % % \end{align*}
    % \end{minipage}
    % \\[2mm]
    % Answer below each question. If you need more space, use the empty space at the end of the exam.
    % \\[-1mm]If needed ask for a new paper.
    % \\[-6mm]
    % }

    \noindent
    \hrulefill
  }

  \newcommand{\newExamPage}{
    \clearpage
    \thispagestyle{empty}
    % ~
    % \newpage
    \begin{center}
      \bfseries
      {\Large #1} (continuation -- \thepage/\pageref{LastPage})\\[2mm]
      DCC-FCUP, University of Porto,
      Jos\'{e} Proen\c{c}a,
      #2
    \end{center}

    % \noindent
    % % \hrulefill
    % % \\[2mm]
    % Number: \rule{20mm}{0.5pt}
    % Name: \rule{116.5mm}{0.5pt}
  }

  \renewcommand{\headrulewidth}{.4mm} % header line width
  \fancyhf{}
  % \fancyhfoffset[L]{1cm} % left extra length
  % \fancyhfoffset[R]{1cm} % right extra length
  \rhead{\thepage/\pageref{LastPage}}
  \lhead{\bfseries #1}
  \rfoot{}
}
%-----------------------------------------


%%% Misc

\def\pv#1#2{\langle #1 \rangle #2}
\def\nc#1#2{[#1]#2}

\newcommand{\dkb}[1]{{\textcolor{blue}{#1}}}
\def\tran#1{\stackrel{#1}{\longrightarrow}}
\def\imp{\mathbin{\Rightarrow}}
\def\comp{\mathbin{\boldsymbol{\cdot}}}


% \newcommand{\note}[1]{
%   \noindent\emph{\color{black!50}\textbf{Note:} #1}
% }

\newcommand{\note}[1]{\[\textcolor{black!60}{
  \!\left[~\begin{minipage}{0.964\textwidth}{\small\it \textbf{Note:} #1}\end{minipage}~\right]
}\]}
%-------------- template --------------------------------------------------
 %!TEX root=../../1-intro.tex
 \usetheme{metropolis}
\usepackage{appendixnumberbeamer}

%------ Setting lecture info ----------------------------------------------
\newcounter{lectureID}
\stepcounter{lectureID}
\newcommand{\getLecture}{\arabic{lectureID}\xspace}
\newcommand{\setLectureBasic}[1]{
  \title{
    #1
    }
  \author{Nelma Moreira ~~\&~~ \textbf{Jos\'{e} Proen\c{c}a}}
  \institute{CISTER -- U.Porto, Porto, Portugal
            \hfill 
            \begin{tabular}{r@{}}
            \url{https://fm-dcc.github.io/pc2324}
            \end{tabular}
            }
  \date{Concurrent programming (CC3040) 2023/2024}
  % logos of institutions
  \titlegraphic{
    \begin{textblock*}{5cm}(2.0cm,7.00cm)
       \includegraphics[scale=0.14]{src/img/logos/fcup}\hspace*{.85cm}~%
    \end{textblock*}
    \begin{textblock*}{5cm}(6.0cm,7.25cm)
      \includegraphics[scale=0.43]{src/img/logos/cmup}
    \end{textblock*}
    \begin{textblock*}{5cm}(10.4cm,7.45cm)
      \includegraphics[scale=0.16]{src/img/logos/cister}
    \end{textblock*}
  }  
  \frame[plain]{\titlepage}
}
\newcommand{\setLecture}[2]{\setcounter{lectureID}{#1}\setLectureBasic{#1. #2}}

%------ Counters for exercises ----------------------------------------------
\newcounter{cExercise}
\newcommand{\exercise}{\stepcounter{cExercise}Ex.\,\arabic{lectureID}.\arabic{cExercise}:\xspace}
\newcommand{\exerciseBack}{\addtocounter{cExercise}{-1}}
\newcommand{\exerciseAdd}{\stepcounter{cExercise}}
\newcommand{\doExercise}[3][0mm]{\begin{exampleblock}{\exercise #2}\wrap{\rule{0pt}{#1}}#3\end{exampleblock}}
\newcommand{\doSimpleExercise}[2][0mm]{\begin{exampleblock}{}\wrap{\rule{0pt}{#1}}\structure{\textbf{\exercise} #2}\end{exampleblock}}

% Slide
\newenvironment{slide}[1]{\begin{frame}\frametitle{#1}}{\end{frame}}



% Base colors (from metropolis theme)
\definecolor{metDarkBrown}{HTML}{604c38}
\definecolor{metDarkTeal}{HTML}{23373b}
\definecolor{metLightBrown}{HTML}{EB811B}
\definecolor{metLightGreen}{HTML}{14B03D}

 

\metroset{numbering=fraction,progressbar=frametitle}

% \setbeamercolor*{structure}{fg=blue!80!black}
\setbeamercolor*{structure}{fg=metLightGreen}

% % \definecolor{MainColour}{rgb}{0., 0.25, 0.8}
% \colorlet{MainColour}{blue!50!black}
% \colorlet{BgColour}{blue!10}
% \colorlet{BarColour}{blue!50!black}

% %\usetheme{CambridgeUS}%{Copenhagen}%{Frankfurt}%{Singapore}%{CambridgeUS}
% \usecolortheme[named=MainColour]{structure} 
% \useoutertheme[subsection=false]{miniframes}
% \useinnertheme{circles}
% %\useinnertheme[shadow=false]{rounded}
% \setbeamertemplate{blocks}[rounded][shadow=false]

% \setbeamercovered{transparent} 
% \setbeamertemplate{navigation symbols}{} %Remove navigation bar
% \setbeamertemplate{footline}[frame number] % add page number
% \setbeamercolor{postit}{fg=MainColour,bg=BgColour}
% \setbeamercolor{structure}{bg=black!10}
% %\setbeamercolor{palette primary}{use=structure,fg=red,bg=green}
% %\setbeamercolor{palette secondary}{use=structure,fg=red!75!black,bg=green}
% \setbeamercolor{palette tertiary}{use=structure,bg=BarColour,fg=white}
% %\setbeamercolor{palette quaternary}{fg=black,bg=green}
% %\setbeamercolor{normal text}{fg=black,bg=white}
% %\setbeamercolor{block title alerted}{fg=red,bg=green}
% %\setbeamercolor{block title example}{bg=black!10,fg=green}
\setbeamercolor{block body}{bg=black!5}

% \setbeamercolor{block title alerted}{bg=red!25}
% \setbeamercolor{block body alerted}{bg=red!10}

% \setbeamercolor{block title example}{bg={rgb:green,2;black,1;white,5}}
\setbeamercolor{block body example}{bg={rgb:green,2;black,1;white,20}}
\setbeamercolor{block body alerted}{bg={metLightBrown!25}}

% \setbeamertemplate{itemize item}{\color{black!10}$\blacksquare$}
\setbeamercolor{itemize item}{fg=metDarkTeal}
\setbeamercolor{itemize subitem}{fg=metDarkTeal}

\setbeamercolor{graybc}{fg=black,bg=black!10}
\newcommand{\myblock}[1]{\begin{beamercolorbox}[dp=1ex,center,rounded=true]%
  {graybc} {\large \textbf{#1}} \end{beamercolorbox}}%

%%%%%%%%%

\definecolor{barcolor}{rgb}{.65,.79,.92} % FCUP color


% Configuring the foot line
\setbeamercolor{author in head/foot}{fg=metDarkTeal, bg=barcolor}%
\setbeamercolor{date in head/foot}{fg=barcolor!75!black}%
\setbeamertemplate{footline}
{
  \leavevmode%
  \hbox{%
  \begin{beamercolorbox}[wd=.4\paperwidth,ht=2.25ex,dp=1ex,center]{author in head/foot}%
    \usebeamerfont{author in head/foot}\insertshortauthor
  \end{beamercolorbox}%
  \begin{beamercolorbox}[wd=.5\paperwidth,ht=2.25ex,dp=1ex,center]{title in head/foot}%
    \usebeamerfont{title in head/foot}\insertsection
  \end{beamercolorbox}%
  \begin{beamercolorbox}[wd=.1\paperwidth,ht=2.25ex,dp=1ex,right]{date in head/foot}%
    \insertframenumber{} / \inserttotalframenumber\hspace*{2ex} 
  \end{beamercolorbox}}%
  \vskip0pt%
}
% No configuration symbols
\setbeamertemplate{navigation symbols}{}


%%%%%%% Custom bar above

\setbeamercolor{frametitle}{bg=barcolor,fg=metDarkTeal}
\setbeamercolor{progress bar}{fg=barcolor!75!black}

%%% Custom bar with a LOGO

\makeatletter
\setbeamertemplate{frametitle}{%
  \nointerlineskip%
  \begin{beamercolorbox}[%
      wd=\paperwidth,%
      sep=0pt,%
      leftskip=\metropolis@frametitle@padding,%
      rightskip=\metropolis@frametitle@padding,%
    ]{frametitle}%
  \metropolis@frametitlestrut@start%
  \insertframetitle%
  \nolinebreak%
  \metropolis@frametitlestrut@end%
  \hfill
  \raisebox{-1.3ex}{\includegraphics[height=4ex,keepaspectratio]{src/img/logos/fcup-bar}}
  \end{beamercolorbox}%
}
\makeatother





%----------------------------------------------------------------------------

\begin{document}

\setLecture{5}{Actor model using the Akka framework}


\section{Overview}

\begin{frame}{We are here}

  \vspace*{-2mm}

  \begin{block}{Blocks of sequential code running concurrently and sharing memory:}
    
  \begin{itemize}
    \item What is Scala?
    \item Concurrency in Java and its memory model
    \item Basic concurrency blocks and libraries
    % \item Futures and promises
    \item \textcolor{gray}{\emph{Futures and Promises}}
    \item \textcolor{gray}{\emph{Data-Parallel Collections}}
    \item \textcolor{gray}{\emph{Reactive Programming (Concurrently)}}
    \item \textcolor{gray}{\emph{Software Transactional Memory}}
    \alert{\item Actor model}
  \end{itemize}
  \end{block}
\end{frame}



% \fromBookW[scale=0.7]{32}{98mm}{39mm}
% \fromBook[scale=0.65]{32}{43mm}{98mm}{43mm}{39mm}

% \begin{frame}[fragile]\frametitle{Current thread}
% ~\\[-8mm]
% \begin{columns}
% \begin{column}{0.49\textwidth}
% \begin{lstlisting}
% ...
% \end{lstlisting}
% \end{column}
% \begin{column}{0.49\textwidth}
% ...
% \end{column}
% \end{columns}
% \end{frame}


\begin{frame}\frametitle{What is the actor model}
  \splittwo{0.4}{0.6}{
    \begin{itemize}
      \item \alert{Asynchronous} message exchange between actors
      \item Introduced in \alert{Erlang}\\(we use Akka's actor library)\\~
      \item<2> Active, autonomous, \alert{no shared memory}, no synchronisation
    \end{itemize}
  }{
    \centering
    \includegraphicsframed[scale=1]{src/img/actor.pdf}\\~\\
    \only<2>{\includegraphicsframed[scale=1]{src/img/actor-com.pdf}}
  }
\end{frame}

\begin{frame}[t]\frametitle{What we will see}

  We will use the Akka framework for actors for:
  \begin{itemize}
    \item Declaring \alert{actor classes} and creating \alert{actor instances}
    \item Modelling \alert{actor state} and complex \alert{actor behaviours}
    \item Manipulating the \alert{actor hierarchy} and the \alert{actor lifecycle}
    \item The different message-passing patterns used in \alert{actor communication}
    \item Error recovery using the built-in \alert{actor supervision} mechanism
    \item Using \alert{remote actors} to build concurrent and distributed programs
  \end{itemize}

  Documentation: \url{https://doc.akka.io/docs/akka}


\end{frame}


\section{Creating actors}

\begin{frame}[t]\frametitle{Core concepts}
  ~\\[-8mm]    
  \splittwo{0.5}{0.5}{
    \begin{exampleblock}{Actor system}
      Hierarchical group of actors with shared configurations, supporting actor creation and logging.
    \end{exampleblock}

    \begin{exampleblock}{Actor class}
      Template that describes the states and behaviour of an actor, used to create instances.
    \end{exampleblock}

    \begin{exampleblock}{Actor instance}
      Entity that exists at runtime, with a state and capable of sending and receiving messages.
    \end{exampleblock}

    \begin{exampleblock}{Message}
      Unit of communication that actors use to communicate. In Akka, any object can be a message.
    \end{exampleblock}
  }{
    \begin{exampleblock}{Mailbox}
      Memory block that is used to buffer messages for a given actor instance.
    \end{exampleblock}

    \begin{exampleblock}{Actor reference}
      Object that allows an object to send messages to a specific actor instance.
    \end{exampleblock}

    \begin{exampleblock}{Dispatcher}
      Component that decides when actors are allowed to process messages. In Akka every dispatcher is also an execution context.
    \end{exampleblock}
  }
\end{frame}

\begin{frame}[fragile]\frametitle{My first actor (class) in Akka}
~\\[-8mm]
\begin{columns}
\begin{column}{0.56\textwidth}
\begin{lstlisting}[emph={Actor, Logging, receive, system,stop}]
import akka.actor._
import akka.event.Logging

class HelloActor(val hello: String) extends Actor {
  val log = Logging(context.system, this)
  def receive = {
    case `hello` =>
      log.info(
        s"Received a '$hello'... $hello!")
    case msg     =>
      log.info(
        s"Unexpected message '$msg'")
      context.stop(self)
  }
}
\end{lstlisting}
\end{column}
\begin{column}{0.44\textwidth}
\begin{itemize}
  \item Each \cod{HelloActor} receives messages
  \item ... if it receives its \cod{hello}, it logs and \structure{continues}
  \item ... if it receives something else, it \structure{stops}
  \item \cod{context} -- provides core functions, such as \cod{stop}
  \item \cod{self} -- is the instance's actor reference
\end{itemize}
\end{column}
\end{columns}
\end{frame}


\begin{frame}[fragile]\frametitle{Configuring an actor in Akka}
~\\[-6mm]
\begin{columns}
\begin{column}{0.51\textwidth}
\begin{lstlisting}[emph={Actor, Logging, Props}]
object HelloActor { // companion
  // two factory methods below
  def props(hello: String) =
    Props(new HelloActor(hello))
  def propsAlt(hello: String) =
    Props(classOf[HelloActor], hello)
  //def propsAlt2 = Props[HelloActor]
}
\end{lstlisting}
\begin{exampleblock}{Actor configuration}
  \begin{itemize}
      \item actor class
      \item constructor arguments
      \item mailbox
      \item dispatcher
  \end{itemize} 
\end{exampleblock}
\end{column}
\begin{column}{0.49\textwidth}
\begin{block}{Props}
  \begin{itemize}
    \item can receive a block of code,
     used each time a new actor instance is created;
     \item can receive a \cod{Class} object and its arguments
     \item can be sent over the network (should be self-contained)
     \item avoid creating \cod{Props} in the actor class, and use factory methods instead
  \end{itemize}
\end{block}
\end{column}
\end{columns}
\end{frame}

\begin{frame}[fragile]\frametitle{My first actor system with an instance}
\begin{lstlisting}[emph={Actor, Logging, Props, ourSystem}]
// in build.sbt:
libraryDependencies ++= Seq( ...
  ,"com.typesafe.akka" %% "akka-actor" % "2.8.5"
  ,"com.typesafe.akka" %% "akka-remote" % "2.8.5"
)
\end{lstlisting}
\begin{lstlisting}[emph={Actor, Logging, Props, ourSystem}]
lazy val ourSystem = akka.actor.ActorSystem("OurExampleSystem")
\end{lstlisting}
\begin{lstlisting}[emph={Actor, Logging, Props, ourSystem,terminate,sleep,actorOf}]
object ActorsCreate extends App {
  val hiActor: ActorRef =
    ourSystem.actorOf(HelloActor.props("ola"), name = "greeter")
  hiActor ! "ola"
  Thread.sleep(1000)
  hiActor ! "hi"
  Thread.sleep(1000)
  ourSystem.terminate()
}
\end{lstlisting}
\end{frame}


\begin{frame}[fragile]\frametitle{Unhandled messages?}

\begin{lstlisting}[emph={Actor, Logging, Props, ourSystem,terminate,sleep,actorOf,receive,unhandled}]
class DeafActor extends Actor {
  val log = Logging(context.system, this)
  def receive = PartialFunction.empty
  override def unhandled(msg: Any) = msg match {
    case msg: String => log.info(s"I do not hear '$msg'")
    case msg         => super.unhandled(msg)
  }
}
\end{lstlisting}
    
\begin{lstlisting}[emph={Actor, Logging, Props, ourSystem,terminate,sleep,actorOf}]
object ActorsUnhandled extends App {
  val deafActor: ActorRef =
    ourSystem.actorOf(Props[DeafActor], name = "deafy")
  deafActor ! "ola"
  Thread.sleep(1000)
  deafActor ! 1234
  Thread.sleep(1000)
  ourSystem.terminate()
}
\end{lstlisting}
\end{frame}


\section{Modelling actor behaviour}

\begin{frame}[fragile]\frametitle{My 2nd example in Akka: a (stateful) countdown}
~\\[-6mm]
\begin{columns}
\begin{column}{0.48\textwidth}
\only<2>{\alert{Not allowed in Akka:}}
\begin{lstlisting}[emph={Actor, Logging, Props, ourSystem,terminate,sleep,actorOf,receive}]
class CountdownActor extends Actor{
  var n = 10
            // never do this
  def receive = if (n > 0) {
    case "count" =>
      log(s"n = $n")
      n -= 1
  } else PartialFunction.empty
}
\end{lstlisting}
\end{column}
\begin{column}{0.52\textwidth}
\pause
\alert{Correct in Akka, using \cod{become}:}
\begin{lstlisting}[emph={Actor, Logging, Props, ourSystem,terminate,sleep,actorOf,receive,become}]
class CountdownActor extends Actor {
  val log = Logging(context.system, this)
  var n = 10
  def counting: Actor.Receive = {
    case "count" =>
      n -= 1
      log.info(s"n = $n")
      if (n == 0) context.become(done)
  }
  def done = PartialFunction.empty
  def receive = counting
}
\end{lstlisting}
\end{column}
\end{columns}
\end{frame}

\begin{frame}\frametitle{Actor as a transition system}
  \centering
  \fromBookW[scale=0.7]{278}{126mm}{43mm}
\end{frame}

\begin{frame}[fragile]\frametitle{Running the countdown}

\begin{lstlisting}[emph={Actor, Logging, Props, ourSystem,terminate,sleep,actorOf,receive,become}]
object ActorsCountdown extends App {
  val countdown = ourSystem.actorOf(Props[CountdownActor])
  for (i <- 0 until 20) countdown ! "count"
  Thread.sleep(1000)
  ourSystem.terminate()
}
\end{lstlisting}


\end{frame}


\section{Actor hierarchy and lifecycle}

\begin{frame}[fragile]\frametitle{New example with a parent}

\begin{columns}
\begin{column}{0.47\textwidth}
~\\[1mm]
\centering
\begin{tikzpicture}
  \tikzstyle{nd}=[rectangle,inner sep=4pt,minimum width=10mm,draw=black!60,fill=barcolor!20,very thick]
  \node[nd,draw=none,fill=none,inner sep=1pt](sys){``OurExampleSystem''};
  \node[nd, below=2mm of sys](par){``parent''};
  \node[nd, below=2mm of par,xshift=-12mm](a){child$_1$};
  \node[nd, below=2mm of par,xshift= 12mm](b){child$_2$};
  \draw[<-,thick] (sys)edge(par) (par)edge(a)edge(b); 
\end{tikzpicture}

\pause
\begin{lstlisting}[emph={Actor, Logging, Props, ourSystem,terminate,context,actorOf,receive}]
class ChildActor extends Actor {
  val log = Logging(context.system, this)
  def receive = {
    case "sayhi" =>
      val parent = context.parent
      log.info(s"my parent $parent made me say hi!")
  }
  override def postStop() {
    log.info("child stopped!")
  }
}
\end{lstlisting}
\end{column}
\begin{column}{0.54\textwidth}
\begin{lstlisting}[emph={Actor, Logging, Props, ourSystem,terminate,sleep,actorOf,receive,become,context}]
class ParentActor extends Actor {
  val log = Logging(context.system, this)
  def receive = {
    case "create" =>
      context.actorOf(Props[ChildActor])
      log.info(s"created a kid; children = ${context.children}")
    case "sayhi" =>
      log.info("Kids, say hi!")
      for (c <- context.children) c ! "sayhi"
    case "stop" =>
      log.info("parent stopping")
      context.stop(self)
  }
}
\end{lstlisting}
\end{column}
\end{columns}
\end{frame}


\begin{frame}[fragile]\frametitle{A more complete view of the hierarchy}
~\\[-6mm]
\begin{columns}
\begin{column}{0.57\textwidth}
\begin{lstlisting}[emph={Actor, Logging, Props, ourSystem,terminate,sleep,actorOf,receive}]
object ActorsHierarchy extends App {
  val parent = ourSystem.actorOf(Props[ParentActor], "parent")
  parent ! "create"
  parent ! "create"
  Thread.sleep(1000)
  parent ! "sayhi"
  Thread.sleep(1000)
  parent ! "stop"
  Thread.sleep(1000)
  ourSystem.terminate()
}\end{lstlisting}
\end{column}
\begin{column}{0.41\textwidth}
  \fromBook[scale=0.6]{284}{46mm}{30mm}{46mm}{143mm}
  ~\\
  \splittwo{0.5}{0.5}{
      - ActorSystem
    \\- sys.terminate
    \\- sys/ctxt.actorOf
    \\
  }{
      - ctxt.stop
    \\- ctxt.become
    \\\alert{- ctxt.children}
    \\\alert{- ctxt.parent}
  }
\end{column}
\end{columns}
\end{frame}


\begin{frame}[fragile]\frametitle{A more complete view of the hierarchy}
~\\[-6mm]
\begin{columns}
\begin{column}{0.57\textwidth}
\begin{itemize}
  \item \alert{parent} actor stops $\Rightarrow$ its \alert{children} stop
  \item \structure{user} and \structure{system}:
    \\ are \alert{guardian actors} -- at the top of the hierarchy, to log, restart actors, etc.
  \item hierarchy visible when printing an actor ref, e.g., for the first child;
    \\\code{akka://OurExampleSystem/user/parent/}\texttt{\$}\code{a}
  \item \pause \alert{Next:}
    \cod{ctxt.actorSelection(path)}
\end{itemize}
\end{column}
\begin{column}{0.41\textwidth}
  \fromBook[scale=0.6]{284}{46mm}{30mm}{46mm}{143mm}
  ~\\
  \splittwo{0.5}{0.5}{
      - ActorSystem
    \\- sys.terminate
    \\- sys/ctxt.actorOf
    \\
  }{
      - ctxt.stop
    \\- ctxt.become
    \\- ctxt.children
    \\- ctxt.parent
  }
\end{column}
\end{columns}
\end{frame}


\begin{frame}[fragile]\frametitle{Discovering actors in the hierarchy}

\begin{lstlisting}[emph={Actor, Logging, Props, ourSystem,terminate,sleep,actorOf,receive,Identify,actorSelection}]
class CheckActor extends Actor {
  val log = Logging(context.system, this)
  def receive = {
    case path: String =>
      log.info(s"checking path $path")
      context.actorSelection(path) ! Identify(path)
    case ActorIdentity(path, Some(ref)) =>
      log.info(s"found actor $ref at $path")
    case ActorIdentity(path, None) =>
      log.info(s"could not find an actor at $path")
  }
}
\end{lstlisting}    

\pause
\begin{lstlisting}[emph={Actor, Logging, Props, ourSystem,terminate,sleep,actorOf,receive,Identify,actorSelection}]
val checker = ourSystem.actorOf(Props[CheckActor], "checker")

checker ! "../*"      // finds the checker and its siblings
checker ! "../../*"   // finds user and system guardians
checker ! "/system/*" // finds internal actors
checker ! "/user/checker2" // logs that no actors were found
\end{lstlisting}
\end{frame}


\begin{frame}\frametitle{Once an actor throws an exception...}
  When an actor throws an exception, a new \alert{``replacement'' actor} is created,
  with the same:
  \begin{itemize}
    \item arguments
    \item mailbox
    \item ActorRef
    \pause
      \begin{itemize}
        \item hence never leak the actual \alert{this} reference!
      \end{itemize}
  \end{itemize}
\end{frame}

\begin{frame}\frametitle{Actor lifecycle}

\begin{itemize}
  \item Created -- \cod{actorOf}
  \item Before starting to process messages -- \cod{preStart()}
  \item After an exception -- \cod{preRestart(t: Throwable, msg: Option[Any])}
    \begin{itemize}
      \item before creating a new actor
      \item when all children are stopped
    \end{itemize}
  \item After recreating a restarted actor -- \cod{postRestart(t: Throwable)}
    \begin{itemize}
      \item the new actor is then assigned the previous mailbox
    \end{itemize}
  \item After an actor terminates -- \cod{postStop()}
    \begin{itemize}
      \item called by the default implementation of \code{preRestart}
    \end{itemize}
\end{itemize}
\end{frame}

\begin{frame}\frametitle{Actor lifecycle in a diagram}
  \centering
  \fromBookW[scale=0.9]{289}{35mm}{130mm}
\end{frame}




\section{Synchrony vs. Asynchrony}

\begin{frame}[t]\frametitle{Sending x and y from A to B}
~\\[-8mm]
\splittwo{0.5}{0.5}{
\begin{exampleblock}{Synchronous (as in CCS)}
$A = x!\,.\,y!$\\
$B = x?\,.\,y?$\\
$A~|~B\backslash\{x,y\}$
\\[4mm]
\pause
$\Rightarrow ~~~~ \tau_x . \tau_y$
\pause
\end{exampleblock}
}{
\begin{alertblock}{Asynchronous (as in Akka)}
$x!$ happens before $y!$\\
$x?$ happens before $y?$\\\pause
$x!$ happens before $x?$\\
$y!$ happens before $y?$\\\pause
$y!$ ?? $x?$
\end{alertblock}
}
~\\
\splittwo{0.5}{0.5}{
\pause
Different formalisations for global beh.:
\begin{itemize}
\item Message sequence charts
\item Event structures
\item Automata over interactions
\item Choreographies:
  \\$A\to B:x~~;~~A\to B:y$
\end{itemize}
}{
  No duplication
  \\ No messages lost
  \\ No messages reordered
  \\ No blocking send
  \\[4mm]
  \structure{Synchrony} modelled with \alert{Asynchrony}?
  \\and vice-versa?
}
\end{frame}

\begin{frame}[t]\frametitle{Diamond problem: sending by two routes}
    
    $A\to B: x;$\\
    $A\to C: y;$\\
    $C\to B: z$

~\\

$B$ must be ready to receive `$x?$' and `$z?$' by any order

\end{frame}



\section{Error recovery with actors}

\begin{frame}[t]\frametitle{Stopping an actor}
\begin{alertblock}{Main ways to stop an actor:}  
\begin{itemize}
  \item \cod{context.stop} -- stops all actors, once they finish processing their current message
  \item \cod{Kill} message -- stops the target actor once it is received
    %once received, will restart the target actor without losing its mailbox
  \item \cod{PoisonPill} message -- stops the target actor \structure{after} processing all the messages currently in its mailbox
\end{itemize}
\end{alertblock}

\pause

\begin{block}{Stopping in more complex scenarios:}
  \begin{itemize}
    \item Using Akka's \alert{DeathWatch} (next slide)
  \end{itemize}
\end{block}
\end{frame}

\begin{frame}[t]\frametitle{Pingy-Pongy example}
% // pongy: pingy?"ping";  pingy!"pong"; stop(self)
% // pingy: master?pongy; pongy!"ping"; pongy?reply; master!reply; loop
% // master: sys?"start"; pingy!pongy; _?"pong"; stop
% master->pingy:pongyRef;
% pingy->pongy:ping;
% pongy->pingy:pong;
% pongy:stop;
% pingy->master:pong;
% master:stop
\centering    
\begin{tabular}{cc}
  \wrap{\includegraphicsframed[width=0.5\textwidth]{src/img/pingy-pongy.pdf}}
  &
  \begin{minipage}{0.3\textwidth}
  \begin{itemize}
    \item Example used in the book to illustrate the ask-reply pattern
    \item (in pingy: \code{val reply = pongy ? "ping"})
    \item We will adapt it for a graceful shutdown
  \end{itemize}
  \end{minipage}
\end{tabular}
\end{frame}

\begin{frame}[fragile]\frametitle{Graceful Pingy-Pongy}
~\\[-6mm]
\begin{columns}[t]
\begin{column}{0.56\textwidth}
% \only<2>{\alert{Not allowed in Akka:}}
\begin{lstlisting}[emph={Actor, Logging, Props, ourSystem,terminate,sleep,actorOf,receive, watch,stop,Terminated}]
class GracefulPingy extends Actor {
  val log = Logging(context.system, this)
  val pongy =
    context.actorOf(Props[Pongy], "pongy")
  context.watch(pongy)
  def receive = {
    case "start" => pongy ! "ping"
    case "pong"  => log.info("Got a pong")
    case "Die, Pingy!" =>
      context.stop(pongy)
    case Terminated(`pongy`) =>
      context.stop(self)
} }
\end{lstlisting}
\end{column}
\begin{column}{0.44\textwidth}
\begin{lstlisting}[emph={Actor, Logging, Props, ourSystem,terminate,sleep,actorOf,receive,become,stop}]
class Pongy extends Actor {
  val log =
    Logging(context.system,this)
  def receive = {
    case "ping" =>
      log.info("Got a ping -- ponging back!")
      sender ! "pong"
  }
  override def postStop() = log.info("pongy going down")
}
\end{lstlisting}
\end{column}
\end{columns}
\end{frame}

\begin{frame}[fragile]\frametitle{Running the gracefull app}
\centering
\splittwo{0.4}{0.6}{
  \includegraphicsframed[width=\textwidth]{src/img/gpingy-pongy.pdf}
}{

  \begin{exampleblock}{Mechanism 1 (pingy $\leftrightarrow$ pongy)}
  \begin{itemize}
    \item \cod{context.watch(pongy)} -- the \code{DeathWatch}
    \item wait for \cod{Terminated} message
  \end{itemize}
  \end{exampleblock}
  \begin{alertblock}{Mechanism 2 (ourSystem $\leftrightarrow$ pingy)}
  \begin{itemize}
    \item ask to \code{``Die''}
    \item check if it terminated -- using Futures
  \end{itemize}
  \end{alertblock}
}

\end{frame}

\begin{frame}[fragile]\frametitle{Running the gracefull app (code)}
~\\[-6mm]
% \begin{columns}[t]
% \begin{column}{0.56\textwidth}
% \begin{lstlisting}[emph={Actor, Logging, Props, ourSystem,terminate,sleep,actorOf,receive, watch,stop}]
%   val log = Logging(context.system, this)
%   val pongy =
%     context.actorOf(Props[Pongy], "pongy")
%   context.watch(pongy)
%   def receive = {
%     case "start" => pongy ! "ping"
%     case "pong"  => log.info("Got a pong")
%     case "Die, Pingy!" =>
%       context.stop(pongy)
%     case Terminated(`pongy`) =>
%       context.stop(self)
% } }
% \end{lstlisting}
% \end{column}
% \begin{column}{0.44\textwidth}
\begin{lstlisting}[emph={Actor, Logging, Props, ourSystem,terminate,sleep,actorOf,receive,become,stop,onComplete,gracefulStop}]
import akka.pattern.gracefulStop

object CommunicatingGracefulStop extends App {
  val gracePingy = ourSystem.actorOf(Props[GracefulPingy], "gracePingy")
  gracePingy ! "start"

  val stopped = gracefulStop(gracePingy, 3.seconds, "Die, Pingy!")
  stopped onComplete { // stopped is a Future (not covered)
    case Success(x) =>
      log("graceful shutdown successful")
      ourSystem.terminate()
    case Failure(t) =>
      log("grace not stopped!")
      ourSystem.terminate()
} }
\end{lstlisting}
% \end{column}
% \end{columns}
\end{frame}


\begin{frame}[fragile]\frametitle{Handling children's exceptions (Actor supervision)}
~\\[-10mm]
\begin{columns}[t]
\begin{column}{0.54\textwidth}
\begin{lstlisting}[emph={Actor, Logging, Props, ourSystem,terminate,sleep,actorOf,receive, watch,stop,Terminated}]

class Naughty extends Actor {
  val log = Logging(context.system,this)
  def receive = {
    case s: String => log.info(s)
    case msg => throw new RuntimeException
  }
  override def postRestart(t:Throwable)=
    log.info("naughty restarted")
}
\end{lstlisting}
\end{column}
\begin{column}{0.50\textwidth}
\begin{lstlisting}[emph={Actor, Logging, Props, ourSystem,terminate,sleep,actorOf,receive,become,stop,supervisorStrategy}]
import SupervisorStrategy._
class Supervisor extends Actor {
  context.actorOf(Props[Naughty], "naughty")
  def receive = PartialFunction.empty
  override val supervisorStrategy =
    OneForOneStrategy() {
      case ake: ActorKilledException => Restart
      case _ => Escalate
} }
\end{lstlisting}
\end{column}
\end{columns}
~\\[-4pt]
\begin{lstlisting}[emph={Actor, Logging, Props, ourSystem,terminate,sleep,actorOf,receive,become,stop}]
ourSystem.actorOf(Props[Supervisor], "super")
val children = ourSystem.actorSelection("/user/super/*")
children ! "hello" // succeeds
children ! Kill    // stops naughty, but super restarts it
children ! "sorry about that" // succeeds
children ! "kaboom".toList    // naughty and super throw exception
\end{lstlisting}
\end{frame}


\section{Remote actors over TCP}

\begin{frame}[fragile]\frametitle{Compilation with remote actors}
~\\[-6.5mm]
\begin{tabular}{ll}
\begin{minipage}{0.18\textwidth}
 \cod{build.sbt} needs to import akka-remote:
\end{minipage}
&
\begin{minipage}{0.82\textwidth}
\begin{lstlisting}[emph={Actor, Logging, Props, ourSystem,terminate,sleep,actorOf,receive,become,stop}]
libraryDependencies ++= Seq(
   ...
  ,"com.typesafe.akka" %% "akka-actor" % "2.8.5" // or older
  ,"com.typesafe.akka" %% "akka-remote" % "2.8.5"
)
\end{lstlisting}
\end{minipage}
\\
\begin{minipage}{0.18\textwidth}
  Network configured with Netty library
\end{minipage}
&
\begin{minipage}{0.82\textwidth}
\begin{lstlisting}[
  showstringspaces=false,
  emph={Actor, Logging, Props, ourSystem,terminate,sleep,actorOf,receive,become,stop,remotingConfig,remotingSystem}]
import com.typesafe.config._
def remotingConfig(port: Int) = ConfigFactory.parseString(s"""
  akka {
    actor.provider = "akka.remote.RemoteActorRefProvider"
    remote {
      enabled-transports = ["akka.remote.netty.tcp"]
      netty.tcp {
        hostname = "127.0.0.1"
        port = $port }
    }
  }""")
def remotingSystem(name: String, port: Int): ActorSystem =
     ActorSystem(name, remotingConfig(port))
\end{lstlisting}
\end{minipage}
% \end{itemize}
\end{tabular}
\end{frame}


\begin{frame}[fragile]\frametitle{Remote Pingy-Pongy -- running two Apps!}
~\\[-6mm]
\begin{columns}[t]
\begin{column}{0.46\textwidth}
\begin{lstlisting}[basicstyle=\ttfamily\tiny,emph={Actor, pingy, pongy, runner, Logging, Props, ourSystem,terminate,sleep,actorOf,receive, watch,stop,Terminated}]
object RemotingPongySystem extends App {
  val system = remotingSystem("PongyDimension", 24321)
  val pongy = system.actorOf(Props[Pongy], "pongy")
  Thread.sleep(15000)
  system.terminate()
}
\end{lstlisting}
\begin{lstlisting}[basicstyle=\ttfamily\tiny,
  emph={Actor, pingy, pongy, runner, Logging, Props, ourSystem,terminate,sleep,actorOf,receive, watch,stop,Terminated}]
object RemotingPingySystem extends App {
  val system = remotingSystem("PingyDimension", 24567)
  val runner = system.actorOf(Props[Runner], "runner")
  runner ! "start"
  Thread.sleep(5000)
  system.terminate()
}
\end{lstlisting}
\end{column}
\begin{column}{0.54\textwidth}
\begin{lstlisting}[basicstyle=\ttfamily\tiny,
  emph={Actor, pingy, pongy, runner, Logging, Props, ourSystem,terminate,sleep,actorOf,receive,become,stop}]
class Runner extends Actor {
  val log = Logging(context.system, this)
  val pingy = context.actorOf(Props[Pingy], "pingy")
  def receive = {
    case "start" =>
      val pongySys = "akka.tcp://PongyDimension@127.0.0.1:24321"
      val pongyPath = "/user/pongy"
      val url = pongySys + pongyPath
      val selection = context.actorSelection(url)
      selection ! Identify(0)
    case ActorIdentity(0, Some(ref)) =>
      pingy ! ref
    case ActorIdentity(0, None) =>
      log.info("Something's wrong - ain't no pongy anywhere!")
      context.stop(self)
    case "pong" =>
      log.info("got a pong from another dimension.")
      context.stop(self)
  }
}
\end{lstlisting}
\end{column}
\end{columns}
\end{frame}

\begin{frame}[t]\frametitle{Running the multi-dimensional Pingy-Pongy}

\begin{itemize}
  \item Start the \cod{RemotingPongySystem}
  \item Start the \cod{RemotingPingySystem} within 15 sec.
  \item Use different SBT instances
  \item Runner in \cod{PingyDimension} should get a ``pong'' soon
\end{itemize}
    
\pause
\begin{block}{Deployment logic vs. Application logic}
  \begin{itemize}
    \item Deployment log.: setting up network communication
    \item Application log.: interactions between agents
    \item These should be kept in separate
    \item In our example, \cod{Runner} handles deployment logic
  \end{itemize}
\end{block}
\end{frame}

\begin{frame}[c]\frametitle{Wrapping up remote actors}
  \begin{exampleblock}{Steps for handling remote actors}    
    \begin{itemize}
      \item \alert{Declaring} each actor system with appropriate remoting configuration
      \item \alert{Starting} each actor system in separate processes or on separate machines
      \item \alert{Obtain actor references} by using actor path selection
      \item \alert{Transparently send messages} by using these actor references
    \end{itemize}
  \end{exampleblock}
\end{frame}


\begin{frame}[t]\frametitle{Wrapping up Actors}
    % We used the Akka framework for actors for:
    \begin{itemize}
      \item Declare \alert{actor classes} and create \alert{actor instances}
      \item Model \alert{actor state} and complex \alert{actor behaviours}
      \item Manipulate the \alert{actor hierarchy} and the \alert{actor lifecycle}
      \item Use some message-passing patterns used in \alert{actor communication}
      \item Use error recovery with the built-in \alert{actor supervision} mechanism
      \item Use \alert{remote actors} to build concurrent and distributed programs
    \end{itemize}

  \splittwo{0.4}{0.6}{
    ~\\~\\
    Documentation: \url{https://doc.akka.io/docs/akka}
  }{
    \includegraphicsframed[scale=1]{src/img/actor-com.pdf}
  }
\end{frame}

\end{document}
