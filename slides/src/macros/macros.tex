\usepackage{transparent}
\usepackage{xspace}
\usepackage{listings}
\usepackage{pdfpages}
\usepackage{relsize}

%%%%%%%%%%%%% Macros
% \newcommand{\Ban}{\catfont{Ban}}
% \newcommand{\Cats}{\catfont{Cat}}

%%%% Misc
%% Operations
\newcommand{\sem}[1]{\llbracket #1 \rrbracket}
\DeclareMathOperator{\img}{\mathrm{im}}
\DeclareMathOperator{\dom}{\mathrm{dom}}
\DeclareMathOperator{\codom}{\mathrm{codom}}
%% Sets of numbers
\newcommand{\N}{\mathbb{N}}
\newcommand{\Z}{\mathbb{Z}}
\newcommand{\Nats}{\mathbb{N}}
\newcommand{\Reals}{\mathbb{R}}
\newcommand{\Rz}{\Reals_{\geq 0}}
\newcommand{\Complex}{\mathbb{C}}
%% Writing
\newcommand{\cf}{\emph{cf.}}
\newcommand{\ie}{\emph{i.e.}}
\newcommand{\eg}{\emph{e.g.}}
\newcommand{\df}[1]{\emph{\textbf{#1}}}
%%%%%%%%%%%%%%%% End of Misc

%%%% Programming Stuff
%% Types
\newcommand{\typefont}[1]{\mathbb{#1}}
\newcommand{\typeOne}{1}
\newcommand{\typeTwo}{2}
\newcommand{\typeA}{\typefont{A}}
\newcommand{\typeX}{\typefont{X}}
\newcommand{\typeB}{\typefont{B}}
\newcommand{\typeC}{\typefont{C}}
\newcommand{\typeV}{\typefont{V}}
\newcommand{\typeD}{\typefont{D}}
\newcommand{\typeI}{\typefont{I}}
%% RuleName
\newcommand{\rulename}[1]{(\mathrm{#1})}
%% Sequents
\newcommand{\jud}{\vdash}
\newcommand{\vljud}{\rhd}
\newcommand{\cojud}{\vdash_{\co}}
\newcommand{\vl}{\mathtt{v}}
\newcommand{\co}{\mathtt{c}}
% Program font
\newcommand{\prog}[1]{\mathtt{#1}}
\newcommand{\pseq}[3]{#1 \leftarrow #2; #3}
\newcommand{\ppm}[4]{(#1,#2) \leftarrow #3; #4}
\newcommand{\pinl}[1]{\prog{inl}(#1)}
\newcommand{\pinr}[1]{\prog{inr}(#1)}
\newcommand{\pcase}[4]{\prog{ case } #1 \prog{ of } \pinl{#2} \Rightarrow #3 ; \pinr{#2} \Rightarrow #4}
%% Sets of terms
\newcommand{\ValuesBP}[2]{\mathsf{Values}(#1, #2)}
\newcommand{\TermsBP}[2]{\mathsf{Terms}(#1, #2)}
\newcommand{\closValP}[1]{\ValuesBP{\emptyset}{#1}}
\newcommand{\closTermP}[1]{\TermsBP{\emptyset}{#1}}
\newcommand{\closVal}{\closValP{\typeA}}
\newcommand{\closTerm}{\closTermP{\typeA}}
%% Contextual equivalence
\newcommand{\ctxeq}{\equiv_{\prog{ctx}}}
%%%% End of Programming Stuff


% Misc by José
\newcommand{\wrap}[2][]{\begin{tabular}[#1]{@{}c@{}}#2\end{tabular}}
\newcommand{\mwrap}[1]{\ensuremath{\begin{array}{@{}c@{}}#1\end{array}}}
\def\trans#1{\xrightarrow{#1}}  % - a - > 
\def\Trans#1{\stackrel{#1}{\Longrightarrow}} % =a=> 
\newcommand{\transp}[2][35]{\color{fg!#1}#2}
\newcommand{\transpt}[2][.35]{\tikz{\node[inner sep=1pt,fill opacity=0.5]{#2}}}
\newcommand{\faded}[2][0.4]{{\transparent{#1}#2}} % alternative to "transp" using transparent package
\newcommand{\set}[1]{\left\{ #1 \right\}} % {a,b,...z}
\newcommand{\mi}[1]{\ensuremath{\mathit{#1}}\xspace}
\newcommand{\mf}[1]{\ensuremath{\mathsf{#1}}\xspace}
% \newcommand{\gold}[1]{\textcolor{darkgoldenrod}{#1}\xspace}


%------ using color ---------------------------------------------------------
\definecolor{goldenrod}{rgb}{.80392 .60784 .11373}
\definecolor{darkgoldenrod}{rgb}{.5451 .39608 .03137}
\definecolor{brown}{rgb}{.15 .15 .15}
\definecolor{darkolivegreen}{rgb}{.33333 .41961 .18431}
\definecolor{myGray}{gray}{0.85}
%
%
\newcommand{\red}[1]{\textcolor{red!80!black}{#1}\xspace}
\newcommand{\blue}[1]{\textcolor{blue}{#1}\xspace}
\newcommand{\gold}[1]{\textcolor{darkgoldenrod}{#1}\xspace}
\newcommand{\gray}[1]{\textcolor{myGray}{#1}\xspace}
% \def\alert#1{{\darkgoldenrod #1}}
% \def\alert#1{{\alert{#1}}}
%\def\brw#1{{\brown #1}}
% \def\structure#1{{\blue #1}}
% \def\tstructure#1{\textbf{\darkblue #1}}
%%\def\gre#1{{\green #1}}
\def\gre#1{{\darkolivegreen #1}}
\def\gry#1{{\textcolor{gray}{#1}}}
\def\rdb#1{{\red #1}}
\def\st{\mathbf{.}\,}
\def\laplace#1#2{*\txt{\mbox{ \fcolorbox{black}{myGray}{$\begin{array}{c}\mbox{#1}\\\\#2\\\\\end{array}$} }}}
%\newcommand{\galois}[2]{#1\; \dashv\; #2}



\def\ainv#1{\overline{#1}}
\def\aconv#1{#1^{\circ}} 
\def\rtran#1{\stackrel{#1}{\longrightarrow}}
\def\pair#1{\langle #1 \rangle}
\def\setdef#1#2{\mathopen{\{} #1 \asor #2 \mathclose{\}}}
\def\imp{\mathbin{\Rightarrow}}
\def\dimp{\mathbin{\Leftrightarrow}}
\def\rimp{\mathbin{\Leftarrow}}
\def\rra{\longrightarrow}
% \def\rcb#1#2#3#4{\def\nothing{}\def\range{#3}\mathopen{\langle}#1 \ #2 \ \ifx\range\nothing::\else: \ #3 :\fi \ #4\mathclose{\rangle}}
\def\abv{\stackrel{\rm abv}{=}}



% Spliting frames in 2 columns
\newcommand{\splittwo}[4]{ 
  \begin{columns}[T]% align columns
  \begin{column}{#1\textwidth} #3 \end{column} ~~~
  \begin{column}{#2\textwidth} #4 \end{column} \end{columns}
}\newcommand{\frsplit}[3][.48]{
  \begin{columns}%[T] % align columns
  \begin{column}{#1\textwidth} #2 \end{column} ~~~
  \begin{column}{#1\textwidth} #3 \end{column} \end{columns}
}
\newcommand{\frsplitdiff}[5][]{
  \begin{columns}[#1]%[T] % align columns
  \begin{column}{#2\textwidth} #4 \end{column} ~~~
  \begin{column}{#3\textwidth} #5 \end{column} \end{columns}
}
\newcommand{\frsplitt}[3][.48]{
  \begin{columns}[T] % align columns
  \begin{column}{#1\textwidth} #2 \end{column} ~~~
  \begin{column}{#1\textwidth} #3 \end{column} \end{columns}
}
\newcommand{\col}[2][.48]{\begin{column}{#1\textwidth} #2 \end{column}}
\newcommand{\colb}[3][.48]{\begin{column}{#1\textwidth} \begin{block}{#2} #3 \end{block} \end{column}}

% Spliting frames in 3 columns
\newcommand{\splitthree}[6]{
  \begin{columns}[T] % align columns
  \begin{column}{#1\textwidth} #4 \end{column} ~~~
  \begin{column}{#2\textwidth} #5 \end{column} ~~~
  \begin{column}{#3\textwidth} #6 \end{column} \end{columns}
}
\newcommand{\frsplitthree}[4][.31]{
  \begin{columns}%[T] % align columns
  \begin{column}{#1\textwidth} #2 \end{column} ~~~
  \begin{column}{#1\textwidth} #3 \end{column} ~~~
  \begin{column}{#1\textwidth} #4 \end{column} \end{columns}
}
\newcommand{\frsplitdiffthree}[7][]{
  \begin{columns}[#1]%[T] % align columns
  \begin{column}{#2\textwidth} #5 \end{column} ~~~
  \begin{column}{#3\textwidth} #6 \end{column} ~~~
  \begin{column}{#4\textwidth} #7 \end{column} \end{columns}
}
\newcommand{\frsplittthree}[4][.32]{
  \begin{columns}[T] % align columns
  \begin{column}{#1\textwidth} #2 \end{column} ~
  \begin{column}{#1\textwidth} #3 \end{column} ~
  \begin{column}{#1\textwidth} #4 \end{column} \end{columns}
}


\newcommand{\typerule}[4][]{\ensuremath{\begin{array}[#1]{c}\textsf{\scriptsize ({#2})} \\#3 \\\hline\raisebox{-3pt}{\ensuremath{#4}}\end{array}}}
\newcommand{\styperule}[3][]{\ensuremath{\begin{array}[#1]{c} #2 \\[0.5mm]\hline\raisebox{-4pt}{\ensuremath{#3}}\end{array}}}
\newcommand{\shrk}{\vspace{-3mm}}

\def\caixa#1{\medskip
  \begin{center}
  \fbox{\begin{minipage}{0.9\textwidth}\protect{#1}\end{minipage}}
  \end{center}}

\newcommand{\mybox}[2][4mm]{
  % \begin{minipage}{#1\textwidth}\begin{block}{}\centering #2\end{block}\end{minipage}}
  \tikz{\node[fill=barcolor!60,align=center,inner sep=#1]{#2};}}
\newcommand{\mycbox}[2][4mm]{
  \begin{center}\mybox[#1]{#2}\end{center}}
  % {\\[-5mm]\centering\mybox[#1]{#2}\\[-5mm]}}

%%%%% Tikz
% \usetikzlibrary{arrows.meta, calc, fit, tikzmark}
\usetikzlibrary{%
  positioning
 ,patterns
 ,arrows
 ,arrows.meta
 ,automata
 ,calc
 ,shapes
 ,fit
 ,tikzmark
 ,fadings
 ,decorations.pathreplacing
 ,plotmarks
% ,pgfplots.groupplots
 ,decorations.markings
}
% \tikzset{shorten >=1pt,node distance=2cm,on grid,auto,initial text={},inner sep=2pt}
\tikzstyle{aut}=[shorten >=1pt,node distance=2cm,on grid,auto,initial text={},inner sep=2pt]
\tikzstyle{st}=[circle,draw=black,fill=black!10,inner sep=3pt]
\tikzstyle{sst}=[rectangle,draw=none,fill=none,inner sep=3pt]
\tikzstyle{final}=[accepting]

%%% Uppaal-like diagrams
\newcommand{\uppbox}[3][20mm]{\tikz{
  \node[black!15,fill=black!15,minimum width=#1,align=left](title){\textbf{{\footnotesize #2}}};
  \node[black!15,fill=black!15,left,xshift=4mm]at(title.east){\textbf{{\footnotesize #2}}};
  \node[blue!60!cyan,right] at(title.west){\textbf{{\footnotesize #2}}};
  \node[below,inner sep=2mm,fill=white,xshift=2mm](box)at(title.south){\includegraphics[width=#1]{#3}};
  \node[fit=(title)(box),draw=black,inner sep=0pt]{};
}}

\newcommand{\uppboxv}[3][20mm]{\tikz{
  \node[below,inner sep=2mm,fill=white](box){\includegraphics[height=#1]{#3}};
  \coordinate[yshift=5mm](top)at(box.north);
  \node[fit=(top)(box.north west)(box.north east),inner sep=0pt,fill=black!15](title){};
  \node[blue!60!cyan,right] at(title.west){\textbf{{\footnotesize #2}}};
  \node[fit=(title)(box),draw=black,inner sep=0pt]{};
}}


%% COnfiguring Listings
\lstset{ % basic style
  language=scala,
  basicstyle=\ttfamily\scriptsize,
  breakatwhitespace=true,
  breaklines=true,
  mathescape,
  % morecomment=[l]{//},
  % morecomment=[n]{/*}{*/},
  % frame=single,                    % adds a frame around the code
  rulecolor=\color{black!40},         % if not set, the frame-color may be changed on line-breaks within not-black text (e.g. comments (green here))
  xleftmargin=1.5mm,
  xrightmargin=1.5mm,
  backgroundcolor=\color{black!5},
  % line numbers
%  numbers=left,  % where to put the line-numbers; possible values are (none, left, right)
 numbersep=5pt, % how far the line-numbers are from the code
 numberstyle=\tiny\color{gray},   
 stepnumber=1,  % the step between two line-numbers. If it is 1 each line will be numbered      
%  xleftmargin=3mm,
%  xrightmargin=1.5mm,
%%%%%
  captionpos=b, % t or b (top or bottom)
  belowcaptionskip=5mm,
%%%%%
  % alsoletter={-},
  % emphstyle=\ttfamily\color{blue}, %\underbar,
  % emphstyle={[2]\ttfamily\color{green!50!black}},
  emphstyle=\bfseries\itshape\color{blue!80!black},       % moreemph={...} - layer keywords
  emphstyle={[2]\itshape\color{red!70!black}},%\underbar} % moreemph={[2]...} - inner keywords
  %
  keywordstyle=\bf\ttfamily\color{red!50!black},
  commentstyle=\sl\ttfamily\color{gray!70},
  % commentstyle=\color{green!60!black},
  stringstyle=\ttfamily\color{green!60!black},
  morestring=[b]",
  % morecomment=[l]{\#},
  frame=single,
  % numberstyle=\tiny, numbers=left, stepnumber=1, firstnumber=1, numberfirstline=true,
  %emph={act,proc,init,sort,map,var,eqn},
  %emph={[2]block,hide,comm,rename,allow,||,<>,sum,&&,=>,true,false},
  literate={\\§}{{{\mbox{\textdollar}}}}1,
  % literate=*{->}{{{\color{red!70!black}$\to$}}}{1}
  %            {.}{{{\color{red!70!black}.}}}{1}
  %            {+}{{{\color{red!70!black}\hspace*{1pt}+\hspace*{1pt}}}}{1}
  %            {|}{{{\color{red!70!black}|}}}{1}
  %            {||}{{{\color{red!70!black}|\!\!|}}}{1}
  %            {*}{{{\color{red!70!black}*}}}{1}
  %            {\#}{{{\color{red!70!black}\#}}}{1}
  %            {&}{{{\color{red!70!black}\&}}}{1}
  %            {:=}{{{\color{red!70!black}:=}}}{1}
  %            {=>}{{{\color{red!70!black}=>}}}{1}
  %            % {>}{{{\color{green!65!black}\hspace*{1pt}>\hspace*{1.5pt}}}}{2}
  %            % {<}{{{\color{green!65!black}\hspace*{1.5pt}<\hspace*{1pt}}}}{2}
  %            % {]}{{{\color{green!65!black}\hspace*{1pt}]\hspace*{1.5pt}}}}{1}
  %            % {[}{{{\color{green!65!black}\hspace*{1.5pt}[\hspace*{1pt}}}}{1}
  %            {>}{{{\color{green!65!black}>}}}{1}
  %            {<}{{{\color{green!65!black}<}}}{1}
  %            {]}{{{\color{green!65!black}]}}}{1}
  %            {[}{{{\color{green!65!black}[}}}{1}
  %  morekeywords={Merger1,Fifo2,Lossy3,Init1,Init2}
  % ,emph={act,proc,init,sort,map,var,eqn}
  % ,emph={[2]block,hide,comm,rename,allow,||,<>,sum,&&,=>}
}

\newcommand{\code}[1]{{\relsize{-1}\ttfamily #1}}
% \definecolor{mydarkgreen}{rgb}{0,0.6,0}
% \newcommand{\bash}[1]{\lstinline[basicstyle=\ttfamily\relsize{-0.5}\color{mydarkgreen},keywordstyle=\bf\sffamily\color{purple},columns=fullflexible,keepspaces,literate=*]|#1|}

\lstdefinestyle{tiny}{basicstyle=\ttfamily\relsize{-7}}

% Include slides from others
\newcommand{\byothers}[3]{{
\begin{frame}{}~\mycbox{\Large slides by #1\\pages #2}\end{frame}{}
\setbeamercolor{background canvas}{bg=}
\includepdf[pages=#2]{../../others/#3}
}}

\newcounter{mypage}
\newcommand{\fromBook}[6][scale=0.5]{
\setcounter{mypage}{20}
\addtocounter{mypage}{#2}
\begin{tabular}{@{}r@{}}
\includegraphics[page=\themypage, trim = #3 #4 #5 #6, clip, #1]%
  {../../learning-concurrent-scala/learning-concurrent-programming-in-scala.pdf}
  \\[0mm]
  \textcolor{myGray}{$\left[\begin{array}{@{}r@{}}
  ~\\[-7mm]
  \text{{\tiny in \emph{``Learning Concurrent}}}\\[-3mm]
  \text{{\tiny \emph{Programming in Scala''}, pg.\,#2}}
  \\[0mm]
  \end{array}\right]$}
\end{tabular}
}

\newcommand{\fromBookW}[4][width=\textwidth]{
\setcounter{mypage}{20}
\addtocounter{mypage}{#2}
\begin{tabular}{@{}r@{}}
\includegraphics[page=\themypage, trim = 20mm #3 20mm #4, clip, #1]%
  {../../learning-concurrent-scala/learning-concurrent-programming-in-scala.pdf}
  \\[-2mm]
  \textcolor{myGray}{$\left[\begin{array}{@{}r@{}}
  ~\\[-7mm]
  \text{{\tiny in \emph{``Learning Concurrent}
                      \emph{Programming in Scala''}, pg.\,#2}}
  \\[0mm]
  \end{array}\right]$}
\end{tabular}
}

\newcommand{\fromAuthor}[3][width=\textwidth]{
\begin{tabular}{@{}r@{}}
\includegraphics[#1]%
  {src/img/#3}
  \\[-2mm]
  \textcolor{myGray}{$\left[\begin{array}{@{}r@{}}
  ~\\[-7mm]
  \text{{\tiny in #2}}
  \\[0mm]
  \end{array}\right]$}
\end{tabular}
}

